%Version 3.1 December 2024
% See section 11 of the User Manual for version history
%
%%%%%%%%%%%%%%%%%%%%%%%%%%%%%%%%%%%%%%%%%%%%%%%%%%%%%%%%%%%%%%%%%%%%%%
%%                                                                 %%
%% Please do not use \input{...} to include other tex files.       %%
%% Submit your LaTeX manuscript as one .tex document.              %%
%%                                                                 %%
%% All additional figures and files should be attached             %%
%% separately and not embedded in the \TeX\ document itself.       %%
%%                                                                 %%
%%%%%%%%%%%%%%%%%%%%%%%%%%%%%%%%%%%%%%%%%%%%%%%%%%%%%%%%%%%%%%%%%%%%%

%%\documentclass[referee,sn-basic]{sn-jnl}% referee option is meant for double line spacing

%%=======================================================%%
%% to print line numbers in the margin use lineno option %%
%%=======================================================%%

%%\documentclass[lineno,pdflatex,sn-basic]{sn-jnl}% Basic Springer Nature Reference Style/Chemistry Reference Style

%%=========================================================================================%%
%% the documentclass is set to pdflatex as default. You can delete it if not appropriate.  %%
%%=========================================================================================%%

%%\documentclass[sn-basic]{sn-jnl}% Basic Springer Nature Reference Style/Chemistry Reference Style

%%Note: the following reference styles support Namedate and Numbered referencing. By default the style follows the most common style. To switch between the options you can add or remove “Numbered” in the optional parenthesis. 
%%The option is available for: sn-basic.bst, sn-chicago.bst%  
 
%%\documentclass[pdflatex,sn-nature]{sn-jnl}% Style for submissions to Nature Portfolio journals
%%\documentclass[pdflatex,sn-basic]{sn-jnl}% Basic Springer Nature Reference Style/Chemistry Reference Style
\documentclass[pdflatex,sn-nature]{sn-jnl}% Math and Physical Sciences Numbered Reference Style
%%\documentclass[pdflatex,sn-mathphys-ay]{sn-jnl}% Math and Physical Sciences Author Year Reference Style
%%\documentclass[pdflatex,sn-aps]{sn-jnl}% American Physical Society (APS) Reference Style
%%\documentclass[pdflatex,sn-vancouver-num]{sn-jnl}% Vancouver Numbered Reference Style
%%\documentclass[pdflatex,sn-vancouver-ay]{sn-jnl}% Vancouver Author Year Reference Style
%%\documentclass[pdflatex,sn-apa]{sn-jnl}% APA Reference Style
%%\documentclass[pdflatex,sn-chicago]{sn-jnl}% Chicago-based Humanities Reference Style

%%%% Standard Packages
%%<additional latex packages if required can be included here>

\usepackage{graphicx}%
\usepackage{multirow}%
\usepackage{amsmath,amssymb,amsfonts}%
\usepackage{amsthm}%
\usepackage{mathrsfs}%
\usepackage[title]{appendix}%
\usepackage{xcolor}%
\usepackage{textcomp}%
\usepackage{manyfoot}%
\usepackage{booktabs}%
\usepackage{algorithm}%
\usepackage{algorithmicx}%
\usepackage{algpseudocode}%
\usepackage{listings}%
\usepackage{bibunits} % for Appendix biblio
\usepackage{lineno}
\defaultbibliography{references} % for appendix
\defaultbibliographystyle{sn-nature} % for appendix

% for table supp
\usepackage{booktabs}
\usepackage{longtable}
\usepackage{colortbl}
\usepackage{xcolor}

\raggedbottom
%%\unnumbered% uncomment this for unnumbered level heads

\begin{document}

\title[Article Title]{Limited net redistribution of marine species during decades of ocean warming}

\author*[1]{\fnm{Federico} \sur{Maioli}}\email{fedma@aqua.dtu.dk} 
\author[1]{\fnm{Daniël} \spfx{van}\sur{Denderen}}\email{pdvd@aqua.dtu.dk} 
\author[2]{\fnm{Max} \sur{Lindmark}}\email{max.lindmark@slu.se}  
\author[1]{\fnm{Marcel} \sur{Montanyès}}\email{m.montanyes@creaf.cat}  
\author[3]{\fnm{Eric\,J.} \sur{Ward}}\email{eric.ward@noaa.gov}  
\author[4]{\fnm{Sean\,C.} \sur{Anderson}}\email{sean.anderson@dfo-mpo.gc.ca}  
\author[1]{\fnm{Martin} \sur{Lindegren}}\email{mli@aqua.dtu.dk} 

\affil*[1]{\orgdiv{National Institute of Aquatic Resources}, \orgname{Technical University of Denmark (DTU)}, \orgaddress{\street{Henrik Dams Allé 202}, \city{Kongens Lyngby}, \postcode{2800}, \country{Denmark}}}  

\affil[2]{\orgdiv{Department of Aquatic Resources}, \orgname{Swedish University of Agricultural Sciences (SLU)}, \orgaddress{\street{Turistgatan 5}, \city{Lysekil}, \postcode{45330}, \country{Sweden}}}  

\affil[3]{\orgdiv{Conservation Biology Division}, \orgname{NOAA Fisheries, Northwest Fisheries Science Center}, \orgaddress{\street{2725 Montlake Blvd E}, \city{Seattle}, \postcode{WA 98112}, \country{USA}}}  

\affil[4]{\orgdiv{Pacific Biological Station}, \orgname{Fisheries and Oceans Canada}, \orgaddress{\street{3190 Hammond Bay Rd}, \city{Nanaimo}, \postcode{BC V9T 6N7}, \country{Canada}}}  

\newpage

\linenumbers

% 200 words
\abstract{Marine species are widely expected to shift poleward or deepward as ocean temperatures rise, tracking the thermal environments required for survival and reproduction. Yet many populations deviate from these expectations, remaining stationary or shifting in variable directions, raising questions about the extent to which marine species reorganize coherently under climate change.

Here we analyse decades of standardized bottom-trawl surveys spanning 11 regions of the North Atlantic and Northeast Pacific to track changes in latitude, longitude, depth, and realized thermal niches of more than 200 demersal fish populations. Across regions, net latitudinal, longitudinal, and depth shifts were generally small, as opposing species trajectories offset one another. In contrast, most populations experienced increasing temperatures within their realized thermal niches, indicating that spatial shifts did not fully offset ocean warming. Within some regions, however, many species shifted coherently along predictable environmental gradients, partially reducing thermal exposure.

Together, these results show that climate-driven redistribution of marine species does not manifest as a simple, cross-regional pattern of poleward and deepward tracking. Instead, responses are regionally structured and multidimensional, highlighting the need for integrative approaches to anticipate biodiversity responses under continued ocean warming.
}

 
\keywords{Marine climate change, Species redistribution, Multidimensional range shifts, Ocean warming, Demersal fishes}

\maketitle

\newpage

\section{Introduction}\label{Introduction}
The world’s oceans are warming rapidly, driving a major reorganization of marine biodiversity \citep{ipcc_ocean_2022}. Marine species are widely expected to respond to ocean warming by tracking their thermal niches---the temperature ranges suitable for survival and reproduction \citep{grinnell_niche-relationships_1917}. Such tracking often involves shifts in geographic distribution \citep{tingley_birds_2009}. Consistent with this expectation, studies across marine taxa and regions document widespread poleward and deepward movements, making these shifts one of the most frequently reported biological responses to climate change \citep{parmesan_globally_2003, parmesan_ecological_2006, poloczanska_global_2013, poloczanska_responses_2016}. These shifts reshape marine ecosystems and influence resource management and conservation planning worldwide \citep{pinsky_preparing_2018, palacios-abrantes_climate_2025, pecl_biodiversity_2017}.

At the same time, observed responses often depart from this general pattern. Many marine populations remain stationary or shift in directions that deviate from poleward or deepward trends, even under sustained warming \citep{rubenstein_climate_2023, fuchs_wrong-way_2020, lawlor_mechanisms_2024}. This variability reflects, in part, the complex spatial structure of ocean warming, including heterogeneous climate velocities and physical constraints imposed by coastlines, bathymetry, and circulation patterns \citep{burrows_pace_2011, pinsky_marine_2013}. As a result, it remains unclear whether heterogeneous species responses add up to a coherent, cross-regional redistribution pattern during ocean warming, or whether opposing movements cancel out at broader spatial scales.

Determining whether species responses combine into coherent redistribution patterns requires evaluating movement across multiple spatial dimensions simultaneously. Most studies, however, examine redistribution along a single gradient at a time---most commonly latitude, or, less often, depth \citep{lenoir_climate-related_2015}. While these approaches have revealed important climate-related patterns, they provide limited insight into how species combine movements along different spatial dimensions, and what this means for net redistribution across regions. In marine systems, species may shift horizontally, move vertically into cooler waters, remain in place while tolerating greater thermal exposure, or combine these strategies \citep{nye_changing_2009}. Without integrating these responses, we lack a complete picture of climate-driven redistribution under ocean warming \citep{lawlor_mechanisms_2024, fredston_reimagining_2025}.

To address this gap, we develop and apply a multidimensional framework that evaluates climate-driven redistribution across three complementary dimensions: horizontal range shifts (changes in range centroids), vertical redistributions (realized depth niches), and changes in realized thermal niches. This framework allows us to assess not only the direction and magnitude of spatial movements, but also whether such movements effectively track shifting thermal environments. If populations fully follow changing isotherms, their realized thermal niches should remain stable through time; systematic warming of realized niches instead indicates incomplete tracking and increasing thermal exposure. We apply this framework to decades of standardized scientific bottom-trawl surveys encompassing more than 200 well-sampled demersal fish populations across 11 regions of the North Atlantic and Northeast Pacific. These regions span broad latitudinal and climatic gradients, include some of the fastest-warming continental shelf seas \citep{pershing_slow_2015, song_arctic_2023}, and contain some of the most comprehensive long-term records of marine biodiversity change (Fig.~\ref{fig:map}; \citep{maureaud_are_2021, maureaud_fishglob_data_2024}). Using spatiotemporal and Bayesian mixed-effects models, we estimate long-term trends in species’ horizontal distributions, depth niches, and realized thermal niches. By examining these responses both within and across regions, we assess whether species-level movements combine into coherent redistribution patterns and evaluate the extent to which such movements reduce thermal exposure under ocean warming.


\begin{figure}[h]
\centering
\includegraphics[width=1\textwidth]{output/figures/main/map.png}
\caption{Overview of survey coverage and bottom temperature trends.
The central map shows the regions included in the analysis, with colored polygons indicating region extents. The tile plot summarizes the number of hauls per region by year. Surrounding panels display regional trends in mean bottom temperature over time; bold values indicate the estimated decadal rate of temperature change.}\label{fig:map}
\end{figure}

\section{Results}\label{Results}

\newcommand{\LonMedian}{-1.45}
\newcommand{\LonCI}{ (95\% CI: -6.76--2.51)}
\newcommand{\LatMedian}{0.63}
\newcommand{\LatCI}{ (95\% CI: -4.27--6.51)}
\newcommand{\DepthMedian}{0.25}
\newcommand{\DepthCI}{ (95\% CI: -0.40--1.03)}
\newcommand{\ThermalMedian}{0.18}
\newcommand{\ThermalCI}{ (95\% CI: 0.07--0.30)}


\subsection{No net spatial shifts, but consistent niche warming}

Overall, we detected no consistent directional shifts in either latitude or longitude across species and regions, as 95\% credible intervals (CIs) greatly overlapped zero (Fig. \ref{fig:posterior_slopes}a). The posterior median global slope ($\beta_{\text{decade}}$) for latitudinal change was \LatMedian{} km~decade$^{-1}$ \LatCI{}, and for longitudinal change \LonMedian{} km~decade$^{-1}$ \LonCI{}. Depth exhibited a slightly stronger, though still uncertain, tendency toward deepening (\DepthMedian{} m~decade$^{-1}$ \DepthCI{}), while thermal niche temperatures increased consistently across species and regions (\ThermalMedian{} $^\circ$C~decade$^{-1}$ \ThermalCI{}).

\begin{figure}[h]
\centering
\includegraphics[width=1\textwidth]{output/figures/main/posterior_slopes.png}
\caption{\label{fig:posterior_slopes}Trends in distributional shifts of demersal fish species, shown from left to right for latitudinal, longitudinal, depth, and thermal niche dimensions. Points and horizontal lines indicate posterior medians and 95\% Bayesian credible intervals of decadal trends.
(a) Global posterior slopes across all regions.
(b) Region-specific posterior slopes.
In (a), interval shading reflects the posterior distribution along a continuous gradient; in (b), color indicates median effect size, with more intense colors representing larger deviations from zero.
Marine regions:  EBS = Eastern Bering Sea, GOA = Gulf of Alaska, BC = British Columbia, USWC = U.S. West Coast,  NEUS-SS = Northeast U.S. and Scotian Shelf, GOM = Gulf of Mexico, BS = Barents Sea, NS = North Sea, CBS = Celtic–Biscay Shelf,  BAL = Baltic Sea, NIC = Northern Iberian Coast.}
\end{figure}

\subsection{Limited average spatial shifts, but coordinated species responses within some regions}

\newcommand{\LonBALMedian}{-1.58}
\newcommand{\LonBALCI}{(95\% CI: -13.05--6.82)}
\newcommand{\LonBCMedian}{0.84}
\newcommand{\LonBCCI}{(95\% CI: -5.60--10.64)}
\newcommand{\LonBSMedian}{-2.35}
\newcommand{\LonBSCI}{(95\% CI: -14.57--5.33)}
\newcommand{\LonCBSMedian}{2.49}
\newcommand{\LonCBSCI}{(95\% CI: -1.02--5.74)}
\newcommand{\LonEBSMedian}{-2.50}
\newcommand{\LonEBSCI}{(95\% CI: -12.35--4.43)}
\newcommand{\LonGOAMedian}{-0.03}
\newcommand{\LonGOACI}{(95\% CI: -7.80--8.11)}
\newcommand{\LonGOMMedian}{-3.30}
\newcommand{\LonGOMCI}{(95\% CI: -13.37--3.41)}
\newcommand{\LonNEUSMedian}{-2.01}
\newcommand{\LonNEUSCI}{(95\% CI: -12.46--5.45)}
\newcommand{\LonNICMedian}{-2.44}
\newcommand{\LonNICCI}{(95\% CI: -9.26--3.25)}
\newcommand{\LonNSMedian}{-4.05}
\newcommand{\LonNSCI}{(95\% CI: -15.69--2.86)}
\newcommand{\LonUSWCMedian}{-0.46}
\newcommand{\LonUSWCCI}{(95\% CI: -5.43--5.00)}
\newcommand{\LatBALMedian}{-2.66}
\newcommand{\LatBALCI}{(95\% CI: -26.20--5.06)}
\newcommand{\LatBCMedian}{-1.74}
\newcommand{\LatBCCI}{(95\% CI: -16.85--5.10)}
\newcommand{\LatBSMedian}{2.43}
\newcommand{\LatBSCI}{(95\% CI: -5.63--19.72)}
\newcommand{\LatCBSMedian}{3.08}
\newcommand{\LatCBSCI}{(95\% CI: -4.44--21.38)}
\newcommand{\LatEBSMedian}{3.60}
\newcommand{\LatEBSCI}{(95\% CI: -2.72--14.41)}
\newcommand{\LatGOAMedian}{0.30}
\newcommand{\LatGOACI}{(95\% CI: -6.11--6.66)}
\newcommand{\LatGOMMedian}{0.33}
\newcommand{\LatGOMCI}{(95\% CI: -2.28--2.95)}
\newcommand{\LatNEUSMedian}{0.69}
\newcommand{\LatNEUSCI}{(95\% CI: -6.01--8.31)}
\newcommand{\LatNICMedian}{-1.28}
\newcommand{\LatNICCI}{(95\% CI: -3.16--0.50)}
\newcommand{\LatNSMedian}{2.00}
\newcommand{\LatNSCI}{(95\% CI: -5.36--14.43)}
\newcommand{\LatUSWCMedian}{0.98}
\newcommand{\LatUSWCCI}{(95\% CI: -8.57--14.87)}
\newcommand{\DepthBALMedian}{-0.02}
\newcommand{\DepthBALCI}{(95\% CI: -0.99--0.74)}
\newcommand{\DepthBCMedian}{0.53}
\newcommand{\DepthBCCI}{(95\% CI: -0.75--3.22)}
\newcommand{\DepthBSMedian}{-0.05}
\newcommand{\DepthBSCI}{(95\% CI: -2.53--1.25)}
\newcommand{\DepthCBSMedian}{0.41}
\newcommand{\DepthCBSCI}{(95\% CI: -0.10--1.01)}
\newcommand{\DepthEBSMedian}{-0.21}
\newcommand{\DepthEBSCI}{(95\% CI: -1.82--0.79)}
\newcommand{\DepthGOAMedian}{0.30}
\newcommand{\DepthGOACI}{(95\% CI: -0.62--1.34)}
\newcommand{\DepthGOMMedian}{-0.11}
\newcommand{\DepthGOMCI}{(95\% CI: -0.76--0.46)}
\newcommand{\DepthNEUSMedian}{0.89}
\newcommand{\DepthNEUSCI}{(95\% CI: -0.10--2.33)}
\newcommand{\DepthNICMedian}{0.14}
\newcommand{\DepthNICCI}{(95\% CI: -0.29--0.58)}
\newcommand{\DepthNSMedian}{0.18}
\newcommand{\DepthNSCI}{(95\% CI: -1.00--1.37)}
\newcommand{\DepthUSWCMedian}{0.86}
\newcommand{\DepthUSWCCI}{(95\% CI: -0.27--2.89)}
\newcommand{\ThermalBALMedian}{0.29}
\newcommand{\ThermalBALCI}{(95\% CI: 0.13--0.44)}
\newcommand{\ThermalBCMedian}{0.30}
\newcommand{\ThermalBCCI}{(95\% CI: 0.23--0.37)}
\newcommand{\ThermalBSMedian}{-0.01}
\newcommand{\ThermalBSCI}{(95\% CI: -0.12--0.11)}
\newcommand{\ThermalCBSMedian}{0.11}
\newcommand{\ThermalCBSCI}{(95\% CI: 0.07--0.15)}
\newcommand{\ThermalEBSMedian}{0.02}
\newcommand{\ThermalEBSCI}{(95\% CI: -0.07--0.14)}
\newcommand{\ThermalGOAMedian}{0.16}
\newcommand{\ThermalGOACI}{(95\% CI: 0.08--0.23)}
\newcommand{\ThermalGOMMedian}{0.24}
\newcommand{\ThermalGOMCI}{(95\% CI: 0.19--0.29)}
\newcommand{\ThermalNEUSMedian}{0.54}
\newcommand{\ThermalNEUSCI}{(95\% CI: 0.46--0.62)}
\newcommand{\ThermalNICMedian}{0.02}
\newcommand{\ThermalNICCI}{(95\% CI: -0.02--0.06)}
\newcommand{\ThermalNSMedian}{0.26}
\newcommand{\ThermalNSCI}{(95\% CI: 0.19--0.33)}
\newcommand{\ThermalUSWCMedian}{0.11}
\newcommand{\ThermalUSWCCI}{(95\% CI: 0.05--0.18)}


\newcommand{\rhocogyc_cogxc_BAL}{0.02 (95\% CI: -0.55--0.58)}
\newcommand{\rhocogxc_cogyc_BAL}{0.02 (95\% CI: -0.55--0.58)}
\newcommand{\rhocogyc_depthnichec_BAL}{-0.34 (95\% CI: -0.89--0.44)}
\newcommand{\rhodepthnichec_cogyc_BAL}{-0.34 (95\% CI: -0.89--0.44)}
\newcommand{\rhocogxc_depthnichec_BAL}{0.40 (95\% CI: -0.31--0.90)}
\newcommand{\rhodepthnichec_cogxc_BAL}{0.40 (95\% CI: -0.31--0.90)}
\newcommand{\rhocogyc_thermalnichec_BAL}{-0.05 (95\% CI: -0.73--0.69)}
\newcommand{\rhothermalnichec_cogyc_BAL}{-0.05 (95\% CI: -0.73--0.69)}
\newcommand{\rhocogxc_thermalnichec_BAL}{-0.40 (95\% CI: -0.90--0.37)}
\newcommand{\rhothermalnichec_cogxc_BAL}{-0.40 (95\% CI: -0.90--0.37)}
\newcommand{\rhodepthnichec_thermalnichec_BAL}{-0.31 (95\% CI: -0.89--0.48)}
\newcommand{\rhothermalnichec_depthnichec_BAL}{-0.31 (95\% CI: -0.89--0.48)}
\newcommand{\rhocogyc_cogxc_BC}{-0.96 (95\% CI: -1.00---0.88)}
\newcommand{\rhocogxc_cogyc_BC}{-0.96 (95\% CI: -1.00---0.88)}
\newcommand{\rhocogyc_depthnichec_BC}{0.04 (95\% CI: -0.38--0.43)}
\newcommand{\rhodepthnichec_cogyc_BC}{0.04 (95\% CI: -0.38--0.43)}
\newcommand{\rhocogxc_depthnichec_BC}{-0.14 (95\% CI: -0.52--0.29)}
\newcommand{\rhodepthnichec_cogxc_BC}{-0.14 (95\% CI: -0.52--0.29)}
\newcommand{\rhocogyc_thermalnichec_BC}{-0.39 (95\% CI: -0.82--0.17)}
\newcommand{\rhothermalnichec_cogyc_BC}{-0.39 (95\% CI: -0.82--0.17)}
\newcommand{\rhocogxc_thermalnichec_BC}{0.47 (95\% CI: -0.09--0.86)}
\newcommand{\rhothermalnichec_cogxc_BC}{0.47 (95\% CI: -0.09--0.86)}
\newcommand{\rhodepthnichec_thermalnichec_BC}{-0.72 (95\% CI: -0.97---0.23)}
\newcommand{\rhothermalnichec_depthnichec_BC}{-0.72 (95\% CI: -0.97---0.23)}
\newcommand{\rhocogyc_cogxc_BS}{-0.20 (95\% CI: -0.62--0.28)}
\newcommand{\rhocogxc_cogyc_BS}{-0.20 (95\% CI: -0.62--0.28)}
\newcommand{\rhocogyc_depthnichec_BS}{-0.04 (95\% CI: -0.57--0.50)}
\newcommand{\rhodepthnichec_cogyc_BS}{-0.04 (95\% CI: -0.57--0.50)}
\newcommand{\rhocogxc_depthnichec_BS}{-0.14 (95\% CI: -0.65--0.41)}
\newcommand{\rhodepthnichec_cogxc_BS}{-0.14 (95\% CI: -0.65--0.41)}
\newcommand{\rhocogyc_thermalnichec_BS}{-0.51 (95\% CI: -0.92--0.17)}
\newcommand{\rhothermalnichec_cogyc_BS}{-0.51 (95\% CI: -0.92--0.17)}
\newcommand{\rhocogxc_thermalnichec_BS}{-0.19 (95\% CI: -0.76--0.47)}
\newcommand{\rhothermalnichec_cogxc_BS}{-0.19 (95\% CI: -0.76--0.47)}
\newcommand{\rhodepthnichec_thermalnichec_BS}{-0.09 (95\% CI: -0.73--0.60)}
\newcommand{\rhothermalnichec_depthnichec_BS}{-0.09 (95\% CI: -0.73--0.60)}
\newcommand{\rhocogyc_cogxc_CBS}{0.29 (95\% CI: -0.19--0.70)}
\newcommand{\rhocogxc_cogyc_CBS}{0.29 (95\% CI: -0.19--0.70)}
\newcommand{\rhocogyc_depthnichec_CBS}{0.20 (95\% CI: -0.45--0.73)}
\newcommand{\rhodepthnichec_cogyc_CBS}{0.20 (95\% CI: -0.45--0.73)}
\newcommand{\rhocogxc_depthnichec_CBS}{-0.50 (95\% CI: -0.91--0.13)}
\newcommand{\rhodepthnichec_cogxc_CBS}{-0.50 (95\% CI: -0.91--0.13)}
\newcommand{\rhocogyc_thermalnichec_CBS}{-0.85 (95\% CI: -0.98---0.52)}
\newcommand{\rhothermalnichec_cogyc_CBS}{-0.85 (95\% CI: -0.98---0.52)}
\newcommand{\rhocogxc_thermalnichec_CBS}{-0.24 (95\% CI: -0.77--0.37)}
\newcommand{\rhothermalnichec_cogxc_CBS}{-0.24 (95\% CI: -0.77--0.37)}
\newcommand{\rhodepthnichec_thermalnichec_CBS}{-0.18 (95\% CI: -0.77--0.53)}
\newcommand{\rhothermalnichec_depthnichec_CBS}{-0.18 (95\% CI: -0.77--0.53)}
\newcommand{\rhocogyc_cogxc_EBS}{-0.46 (95\% CI: -0.79--0.02)}
\newcommand{\rhocogxc_cogyc_EBS}{-0.46 (95\% CI: -0.79--0.02)}
\newcommand{\rhocogyc_depthnichec_EBS}{-0.14 (95\% CI: -0.59--0.37)}
\newcommand{\rhodepthnichec_cogyc_EBS}{-0.14 (95\% CI: -0.59--0.37)}
\newcommand{\rhocogxc_depthnichec_EBS}{-0.50 (95\% CI: -0.82---0.05)}
\newcommand{\rhodepthnichec_cogxc_EBS}{-0.50 (95\% CI: -0.82---0.05)}
\newcommand{\rhocogyc_thermalnichec_EBS}{-0.63 (95\% CI: -0.94---0.01)}
\newcommand{\rhothermalnichec_cogyc_EBS}{-0.63 (95\% CI: -0.94---0.01)}
\newcommand{\rhocogxc_thermalnichec_EBS}{0.11 (95\% CI: -0.48--0.66)}
\newcommand{\rhothermalnichec_cogxc_EBS}{0.11 (95\% CI: -0.48--0.66)}
\newcommand{\rhodepthnichec_thermalnichec_EBS}{0.49 (95\% CI: -0.15--0.88)}
\newcommand{\rhothermalnichec_depthnichec_EBS}{0.49 (95\% CI: -0.15--0.88)}
\newcommand{\rhocogyc_cogxc_GOA}{0.52 (95\% CI: -0.26--0.94)}
\newcommand{\rhocogxc_cogyc_GOA}{0.52 (95\% CI: -0.26--0.94)}
\newcommand{\rhocogyc_depthnichec_GOA}{-0.12 (95\% CI: -0.81--0.71)}
\newcommand{\rhodepthnichec_cogyc_GOA}{-0.12 (95\% CI: -0.81--0.71)}
\newcommand{\rhocogxc_depthnichec_GOA}{0.06 (95\% CI: -0.72--0.81)}
\newcommand{\rhodepthnichec_cogxc_GOA}{0.06 (95\% CI: -0.72--0.81)}
\newcommand{\rhocogyc_thermalnichec_GOA}{0.10 (95\% CI: -0.76--0.84)}
\newcommand{\rhothermalnichec_cogyc_GOA}{0.10 (95\% CI: -0.76--0.84)}
\newcommand{\rhocogxc_thermalnichec_GOA}{0.07 (95\% CI: -0.79--0.84)}
\newcommand{\rhothermalnichec_cogxc_GOA}{0.07 (95\% CI: -0.79--0.84)}
\newcommand{\rhodepthnichec_thermalnichec_GOA}{0.02 (95\% CI: -0.80--0.82)}
\newcommand{\rhothermalnichec_depthnichec_GOA}{0.02 (95\% CI: -0.80--0.82)}
\newcommand{\rhocogyc_cogxc_GOM}{0.90 (95\% CI: 0.74--0.98)}
\newcommand{\rhocogxc_cogyc_GOM}{0.90 (95\% CI: 0.74--0.98)}
\newcommand{\rhocogyc_depthnichec_GOM}{0.22 (95\% CI: -0.19--0.60)}
\newcommand{\rhodepthnichec_cogyc_GOM}{0.22 (95\% CI: -0.19--0.60)}
\newcommand{\rhocogxc_depthnichec_GOM}{0.49 (95\% CI: 0.12--0.77)}
\newcommand{\rhodepthnichec_cogxc_GOM}{0.49 (95\% CI: 0.12--0.77)}
\newcommand{\rhocogyc_thermalnichec_GOM}{-0.22 (95\% CI: -0.67--0.27)}
\newcommand{\rhothermalnichec_cogyc_GOM}{-0.22 (95\% CI: -0.67--0.27)}
\newcommand{\rhocogxc_thermalnichec_GOM}{-0.42 (95\% CI: -0.79--0.06)}
\newcommand{\rhothermalnichec_cogxc_GOM}{-0.42 (95\% CI: -0.79--0.06)}
\newcommand{\rhodepthnichec_thermalnichec_GOM}{-0.79 (95\% CI: -0.98---0.38)}
\newcommand{\rhothermalnichec_depthnichec_GOM}{-0.79 (95\% CI: -0.98---0.38)}
\newcommand{\rhocogyc_cogxc_NEUS-SS}{0.86 (95\% CI: 0.68--0.96)}
\newcommand{\rhocogxc_cogyc_NEUS-SS}{0.86 (95\% CI: 0.68--0.96)}
\newcommand{\rhocogyc_depthnichec_NEUS-SS}{-0.21 (95\% CI: -0.59--0.22)}
\newcommand{\rhodepthnichec_cogyc_NEUS-SS}{-0.21 (95\% CI: -0.59--0.22)}
\newcommand{\rhocogxc_depthnichec_NEUS-SS}{-0.29 (95\% CI: -0.64--0.12)}
\newcommand{\rhodepthnichec_cogxc_NEUS-SS}{-0.29 (95\% CI: -0.64--0.12)}
\newcommand{\rhocogyc_thermalnichec_NEUS-SS}{-0.88 (95\% CI: -0.99---0.66)}
\newcommand{\rhothermalnichec_cogyc_NEUS-SS}{-0.88 (95\% CI: -0.99---0.66)}
\newcommand{\rhocogxc_thermalnichec_NEUS-SS}{-0.84 (95\% CI: -0.97---0.57)}
\newcommand{\rhothermalnichec_cogxc_NEUS-SS}{-0.84 (95\% CI: -0.97---0.57)}
\newcommand{\rhodepthnichec_thermalnichec_NEUS-SS}{0.26 (95\% CI: -0.20--0.65)}
\newcommand{\rhothermalnichec_depthnichec_NEUS-SS}{0.26 (95\% CI: -0.20--0.65)}
\newcommand{\rhocogyc_cogxc_NIC}{0.37 (95\% CI: -0.24--0.84)}
\newcommand{\rhocogxc_cogyc_NIC}{0.37 (95\% CI: -0.24--0.84)}
\newcommand{\rhocogyc_depthnichec_NIC}{0.06 (95\% CI: -0.74--0.77)}
\newcommand{\rhodepthnichec_cogyc_NIC}{0.06 (95\% CI: -0.74--0.77)}
\newcommand{\rhocogxc_depthnichec_NIC}{-0.35 (95\% CI: -0.89--0.52)}
\newcommand{\rhodepthnichec_cogxc_NIC}{-0.35 (95\% CI: -0.89--0.52)}
\newcommand{\rhocogyc_thermalnichec_NIC}{-0.12 (95\% CI: -0.86--0.75)}
\newcommand{\rhothermalnichec_cogyc_NIC}{-0.12 (95\% CI: -0.86--0.75)}
\newcommand{\rhocogxc_thermalnichec_NIC}{-0.12 (95\% CI: -0.86--0.77)}
\newcommand{\rhothermalnichec_cogxc_NIC}{-0.12 (95\% CI: -0.86--0.77)}
\newcommand{\rhodepthnichec_thermalnichec_NIC}{0.02 (95\% CI: -0.81--0.83)}
\newcommand{\rhothermalnichec_depthnichec_NIC}{0.02 (95\% CI: -0.81--0.83)}
\newcommand{\rhocogyc_cogxc_NS}{-0.52 (95\% CI: -0.80---0.11)}
\newcommand{\rhocogxc_cogyc_NS}{-0.52 (95\% CI: -0.80---0.11)}
\newcommand{\rhocogyc_depthnichec_NS}{0.86 (95\% CI: 0.62--0.97)}
\newcommand{\rhodepthnichec_cogyc_NS}{0.86 (95\% CI: 0.62--0.97)}
\newcommand{\rhocogxc_depthnichec_NS}{-0.60 (95\% CI: -0.85---0.22)}
\newcommand{\rhodepthnichec_cogxc_NS}{-0.60 (95\% CI: -0.85---0.22)}
\newcommand{\rhocogyc_thermalnichec_NS}{-0.81 (95\% CI: -0.98---0.43)}
\newcommand{\rhothermalnichec_cogyc_NS}{-0.81 (95\% CI: -0.98---0.43)}
\newcommand{\rhocogxc_thermalnichec_NS}{0.24 (95\% CI: -0.31--0.71)}
\newcommand{\rhothermalnichec_cogxc_NS}{0.24 (95\% CI: -0.31--0.71)}
\newcommand{\rhodepthnichec_thermalnichec_NS}{-0.69 (95\% CI: -0.95---0.23)}
\newcommand{\rhothermalnichec_depthnichec_NS}{-0.69 (95\% CI: -0.95---0.23)}
\newcommand{\rhocogyc_cogxc_USWC}{-0.78 (95\% CI: -0.96---0.44)}
\newcommand{\rhocogxc_cogyc_USWC}{-0.78 (95\% CI: -0.96---0.44)}
\newcommand{\rhocogyc_depthnichec_USWC}{-0.49 (95\% CI: -0.87--0.01)}
\newcommand{\rhodepthnichec_cogyc_USWC}{-0.49 (95\% CI: -0.87--0.01)}
\newcommand{\rhocogxc_depthnichec_USWC}{0.22 (95\% CI: -0.36--0.74)}
\newcommand{\rhodepthnichec_cogxc_USWC}{0.22 (95\% CI: -0.36--0.74)}
\newcommand{\rhocogyc_thermalnichec_USWC}{-0.21 (95\% CI: -0.86--0.67)}
\newcommand{\rhothermalnichec_cogyc_USWC}{-0.21 (95\% CI: -0.86--0.67)}
\newcommand{\rhocogxc_thermalnichec_USWC}{0.23 (95\% CI: -0.68--0.90)}
\newcommand{\rhothermalnichec_cogxc_USWC}{0.23 (95\% CI: -0.68--0.90)}
\newcommand{\rhodepthnichec_thermalnichec_USWC}{0.05 (95\% CI: -0.77--0.81)}
\newcommand{\rhothermalnichec_depthnichec_USWC}{0.05 (95\% CI: -0.77--0.81)}


At the regional scale, posterior slopes for spatial shifts varied in direction but were generally small and uncertain, offering limited evidence for consistent changes in mean latitude, longitude, or depth across species (Fig.~\ref{fig:posterior_slopes}b). Depth showed the clearest---though still weak and variable---signals, with modest deepening along the U.S. West Coast (USWC; \DepthUSWCMedian{}~m~decade$^{-1}$ \DepthUSWCCI{}) and the Northeast U.S. and Scotian Shelf (NEUS–SS; \DepthNEUSMedian{}~m~decade$^{-1}$ \DepthNEUSCI{}). Although regional slopes were weak, many species within some regions shifted in the same direction (Fig.~\ref{fig:prop_significant}). In the Eastern Bering Sea (EBS), 64\% of species moved northward, with a similar proportion in the North Sea (NS; 60\%), whereas six of eight species in the Baltic Sea (BAL) shifted southward. Depth shifts were also directionally consistent within some regions, with many species deepening in the NEUS–SS (39\%) and British Columbia (BC; 34\%), while few moved shallower. In contrast, some regions showed no detectable spatial shifts---for example, no species in the Gulf of Alaska (GOA) shifted in latitude, longitude, or depth, and depth did not change in the Northern Iberian Coast (NIC). Thermal-niche temperatures increased in nearly all regions and closely tracked local ocean-warming rates (Fig.~\ref{fig:map}; \textit{$\rho$} = 0.79, \textit{n} = 11, \textit{p} = 0.0036). The only regions without widespread thermal-niche warming were the EBS, Barents Sea (BS), and NIC.

Although thermal niche warming was widespread, in some regions species that shifted poleward or deeper experienced smaller increases in their thermal-niche temperatures. Negative correlations between latitudinal shifts and thermal niche warming were strongest in the NEUS–SS ($\rho$ = \rhocogycthermalnichecNEUS), CBS (\rhocogycthermalnichecCBS), NS (\rhocogycthermalnichecNS), and EBS (\rhocogycthermalnichecEBS), while depth shifts showed similar patterns in the GOM (\rhodepthnichecthermalnichecGOM), BC (\rhodepthnichecthermalnichecBC), and NS (\rhodepthnichecthermalnichecNS). Longitudinal shifts generally showed weak or no relationship with thermal-niche change, except in the NEUS–SS, where longitudinal movements co-varied with latitude (\rhocogyccogxcNEUS). Spatial shifts along different axes also co-varied in other regions: in BC, USWC, NEUS–SS, and GOM, latitudinal and longitudinal shifts were associated, reflecting coast-parallel movements, whereas in the NS, latitudinal and depth shifts were correlated, with northward movements coinciding with deepening (\rhodepthnicheccogycNS).

\begin{figure}[H]
\centering
\includegraphics[width=1\textwidth]{output/figures/main/prop_significant.png}
\caption{\label{fig:prop_significant}Proportion of species with supported distributional shifts over time. Trends are shown for latitudinal, longitudinal, depth, and thermal niche shifts. Bar plots indicate the proportion of species per region showing a particular response, with shifts considered supported if at least 95\% of the posterior distribution excludes zero; otherwise, shifts are marked as non-significant (n.s.). Marine regions: EBS = Eastern Bering Sea, GOA = Gulf of Alaska, BC = British Columbia, USWC = U.S. West Coast,  NEUS-SS = Northeast U.S. and Scotian Shelf, GOM = Gulf of Mexico, BS = Barents Sea, NS = North Sea, CBS = Celtic–Biscay Shelf,  BAL = Baltic Sea, NIC = Northern Iberian Coast.
}
\end{figure}

\subsection{Species show heterogeneous spatial shifts but niche warming is widespread}

\newcommand{\maxlat_shift_species}{\textit{Molva molva}}
\newcommand{\maxlat_shift_region}{CBS}
\newcommand{\max_lat_shift_value}{194.09 (95\% CI: 142.02--239.83)}
\newcommand{\min_lat_shift_species}{\textit{Atheresthes stomias}}
\newcommand{\min_lat_shift_region}{COW}
\newcommand{\min_lat_shift_value}{-101.57 (95\% CI: -133.19---71.11)}
\newcommand{\maxlon_shift_species}{\textit{Harengula jaguana}}
\newcommand{\maxlon_shift_region}{GMX}
\newcommand{\max_lon_shift_value}{116.42 (95\% CI: 90.01--142.20)}
\newcommand{\min_lon_shift_species}{\textit{Squalus acanthias}}
\newcommand{\min_lon_shift_region}{NEUS}
\newcommand{\min_lon_shift_value}{-107.64 (95\% CI: -138.02---76.61)}
\newcommand{\maxdepth_shift_species}{\textit{Squalus suckleyi}}
\newcommand{\maxdepth_shift_region}{BC}
\newcommand{\max_depth_shift_value}{32.10 (95\% CI: 24.76--39.56)}
\newcommand{\min_depth_shift_species}{\textit{Anoplopoma fimbria}}
\newcommand{\min_depth_shift_region}{BC}
\newcommand{\min_depth_shift_value}{-15.94 (95\% CI: -37.61--1.24)}
\newcommand{\maxthermal_shift_species}{\textit{Squalus acanthias}}
\newcommand{\maxthermal_shift_region}{NEUS}
\newcommand{\max_thermal_shift_value}{1.11 (95\% CI: 0.91--1.32)}
\newcommand{\min_thermal_shift_species}{\textit{Atheresthes stomias}}
\newcommand{\min_thermal_shift_region}{EBS}
\newcommand{\min_thermal_shift_value}{-0.17 (95\% CI: -0.32---0.03)}

\newcommand{\LatPerc}{45\%}
\newcommand{\LonPerc}{40\%}
\newcommand{\DepthPerc}{26\%}
\newcommand{\ThermalPerc}{67\%}

\newcommand{\LatNorthingPerc}{24\%}
\newcommand{\LatSouthingPerc}{21\%}
\newcommand{\LonEastingPerc}{17\%}
\newcommand{\LonWestingPerc}{24\%}
\newcommand{\DepthDeepeningPerc}{16\%}
\newcommand{\DepthShallowingPerc}{13\%}
\newcommand{\ThermalCoolingPerc}{0\%}
\newcommand{\ThermalWarmingPerc}{62\%}


At the species level, spatial shifts were highly variable in both direction and magnitude (Fig.~\ref{fig:prop_significant}, Supplementary Fig.~\ref{fig:posterior_slopes_supp}). About \LatPerc{} of the 226 populations shifted in latitude, with northward and southward movements occurring in similar proportions. The most extreme northward shift was seen in \MaxLatSpecies{} in \MaxLatRegion{} (\MaxLatMedian{}~km~decade$^{-1}$ \MaxLatCI{}), and the largest southward in \MinLatSpecies{}~km~decade$^{-1}$ in \MinLatRegion{} (\MinLatMedian{} \MinLatCI{}). Longitudinal shifts occurred in 40\% of species, with \LonWestingPerc{} moving westward and \LonEastingPerc{} eastward; the strongest eastward and westward shifts were \MaxLonSpecies{} in \MaxLonRegion{} (\MaxLonMedian{}~km~decade$^{-1}$ \MaxLonCI{}) and \MinLonSpecies{} in \MinLonRegion{} (\MinLonMedian{}~km~decade$^{-1}$ \MinLonCI{}), respectively. Depth shifts were less frequent, with \DepthDeepeningPerc{} of species moving deeper and \DepthShallowingPerc{} shallower. The strongest deepening occurred in \MaxDepthSpecies{} in \MaxDepthRegion{} (\MaxDepthMedian{}~m~decade$^{-1}$ \MaxDepthCI{}), and the greatest shallowing in \MinDepthSpecies{} in \MinDepthRegion{} (\MinDepthMedian{}~m~decade$^{-1}$  \MinDepthCI{}). 
In contrast, most species experienced warming in their thermal niches (\ThermalPerc{}), led by \MaxThermalSpecies{} in \MaxThermalRegion{} (\MaxThermalMedian{}~$^\circ$C~decade$^{-1}$ \MaxThermalCI{}), and only three species showing signs of cooling.

%\subsection{Species poorly track climate velocities}

%Overall, species did not consistently track either the magnitude or the direction of their predicted thermal envelope shifts. Correlations between observed and expected shifts were generally weak, and only about half of the species exhibited alignment in shift direction (Supplementary Table \ref{tab:thermal_envelope}). This pattern was consistent across all regions, with no significant positive correlations observed between expected and actual shifts (Supplementary Table \ref{tab:thermal_envelope_regions}). However, some regions showed higher directional agreement: for example, 79\% of species in the EBS and 69\% in the BS displayed matching latitudinal shift directions between observed and expected values (Supplementary Table \ref{tab:thermal_envelope_regions}).



\section{Discussion}\label{sec:Discussion}

Marine biodiversity is undergoing rapid reorganization under ocean warming, with many studies documenting poleward and deepward range shifts as species track their thermal niches \citep{parmesan_globally_2003, perry_climate_2005, dulvy_climate_2008}. However, these latitudinal and vertical redistributions capture only a subset of possible responses and are typically examined in isolation \citep{fredston_reimagining_2025}. This one-dimensional perspective constrains our understanding of how species respond across the full suite of ecological pathways available to them—and whether such movements effectively reduce thermal exposure. To address this gap, we developed a multidimensional framework that jointly quantifies horizontal, vertical, and thermal niche trends for over 200 bottom-dwelling fish populations in 11 continental shelf regions spanning different level of warming. Our results show that redistributions are highly variable, weakly coordinated, and often ineffective at buffering populations from warming but with exceptions.

Many global syntheses have reported net poleward migration---typically on the order of tens of kilometers per decade---but our findings differ substantially. For example, Poloczanska et al. \cite{poloczanska_global_2013} estimated mean poleward shifts of 72.0 $\pm$ 13.5 km decade$^{-1}$ at species’ leading edges and 30.6 $\pm$  5.2 km decade$^{-1}$ for centroids, while Lenoir et al. \cite{lenoir_species_2020} reported comparable rates of 59.2 $\pm$ 9.4 km decade$^{-1}$ in the BioShifts database. In contrast, our median latitudinal shift of \LatMedian{} km decade$^{-1}$ \LatCI{} was one to two orders of magnitude smaller, with northward and southward movements largely balancing out across species and regions. Similar to \citep{rubenstein_climate_2023} we also found little evidence for a consistent deepward redistribution across regions, with a median depth shift of \DepthMedian{} m decade$^{-1}$ (\DepthCI). These results indicate that no single axis of movement dominates marine redistribution, contradicting expectations of uniform poleward or downward tracking.

The absence of coherent shifts does not imply a lack of biological response. Rather, it reflects how physical and biological processes jointly shape redistribution pathways. At broad scales, temperature declines predictably with latitude and depth, but local climate velocities, bathymetric barriers, currents, hypoxia, and habitat structure generate complex mosaics of thermal opportunity (e.g., \citep{pinsky_marine_2013, garcia_molinos_climate_2016}). Species traits, dispersal limits, habitat dependence, and non-thermal factors---such as fishing, oxygen or prey availability---further modulate responses, often decoupling movement from simple thermal tracking \citep{sunday_species_2015,deutsch_climate_2015,wisz_role_2013,van_der_putten_predicting_2010}. These interacting constraints produce heterogeneous, sometimes opposing, shifts---even among co-occurring species.

These physical and biological constraints help explain why redistributions only weakly tracked expected thermal-envelope shifts. Roughly half of populations moved in the direction predicted by local warming, while the remainder shifted oppositely or exhibited little movement. In many cases, the magnitude of observed shifts exceeded that of thermal gradients, suggesting that dispersal ability is not the key limiting factor. Yet despite these shifts, most populations continued to warm roughly in step with regional warming trends. The resulting mismatch between redistribution and thermal niche change indicates that redistribution rarely provides full thermal compensation. Marine species appear to occupy dynamic thermal niches to some extent, rather than simply tracking static isotherms \citep{fredston_range_2021, ward_win_2024}. This partial compensation suggests that behavioral or distributional responses may not be sufficient to buffer many populations from continued warming, potentially increasing their exposure to thermal stress. % some citation about lagging behind I guess

Not all regions or species deviate from expected redistribution patterns. In the Eastern Bering Sea, for example, 88\% of species shifted in the thermally predicted latitudinal direction, with three times more species moving north than south. On the U.S. West Coast and in British Columbia, vertical movements dominated, consistent with depth-temperature gradients. The North Sea stands out as a case of highly coherent adjustment \cite{perry_climate_2005}: more than 60\% of species shifted in predicted directions along latitude, depth, and longitude, with correlations exceeding $\rho > 0.8$. At the species level, North Sea cod (\textit{Gadus morhua})---a well studied example \citep{perry_climate_2005, engelhard_climate_2014}---shifted simultaneously northward and deeper, effectively tracking its thermal niche. These cases illustrate that coherent thermal tracking can occur, but it is not the norm across marine systems.

Partial thermal compensation emerged within several regions through movements along the locally dominant environmental gradients. Where strong latitudinal gradients existed---such as in the Eastern Bering Sea, NEUS–SS, and parts of the North Sea---poleward movements reduced thermal exposure. In regions with limited latitudinal gradient, such as the Gulf of Mexico, compensation instead relied on depth. Depth also became the effective axis where horizontal and vertical gradients were tightly coupled, as in much of the North Sea, or where steep vertical gradients were available, as in British Columbia.

Many cross-regional analyses pool data from heterogeneous sources and methodologies and typically lack uncertainty estimates for underlying shifts \citep{brown_ecological_2016, lenoir_species_2020, dahms_temperature_2023}. Our two-step modeling framework addresses these limitations within a unified statistical structure. First, we fitted spatiotemporal models directly to raw survey data to estimate distributional metrics and their uncertainties while accounting for uneven sampling effort and methodological changes over time \citep{thorson_model-based_2016}. Second, we modeled temporal trends in these metrics within a Bayesian framework that propagated uncertainty consistently from the observation process to species-, regional-, and cross-regional summaries. This approach reduces methodological inconsistency and enables formal cross-system comparisons. The trade-off is reduced taxonomic and geographic coverage, as analyses were necessarily limited to well-sampled species in standardized surveys. These methodological differences likely contribute to the smaller and more heterogeneous shift rates we report compared with earlier cross-regional syntheses \citep{poloczanska_global_2013, lenoir_species_2020}. % also something about the fact that is multidimensional!!!

Taken together, our results show that marine redistributions are highly heterogeneous, weakly coordinated, and only partially aligned with thermal change. The limited cross-regional coherence underscores that range shifts are context-dependent processes shaped by the interplay of environmental gradients, species traits, habitat structure, and ecological interactions. This complexity challenges the simplistic assumption that species universally track cooler habitats along predictable gradients and highlights the need for multidimensional approaches to forecast biodiversity responses under ocean warming.


%Together, our findings reveal that marine redistributions under warming are not globally coherent adjustments along single axes but multidimensional, context-dependent responses shaped by local physical constraints and species-specific ecological pathways. While some populations track warming effectively, many show limited or misaligned shifts, resulting in persistent thermal exposure. These patterns suggest fundamental limits to distributional adaptation under continued ocean warming and highlight the need for climate-ready management strategies that account for regional complexity, response diversity, and the limited buffering capacity of redistribution.



%hiearachical scale with global, regional and species tithin regions allowed us to tease apart  broad- and fine-scale patterns and to address common sources of  error in studying range shifts (Brown et al., 2016). For example, measuring edge shifts only using latitude and not using our alongshore  distance metric might have masked some range shifts that occurred  in sections of the coast that slant eastward. We avoided some of the most common sources of  methodological error in detecting range shifts using a long timeseries, focusing on abundant species, and calculating edge position  as a percentile rather than from the most extreme values (Bates  et al., 2015; Brown et al., 2016). However, the high degree of variability that we observed in species’ range edge dynamics may be  as much or more due to imperfect sampling than to actual fluctuations in the species range.




%These within-region correlations show that, although global coherence is weak, local movements can still mitigate warming exposure along available thermal pathways. 





%Analyzing our data at.. allowed us to tease apart  broad- and fine-scale patterns and to address common sources of  error in studying range shifts (Brown et al., 2016). For example, measuring edge shifts only using latitude and not using our alongshore  distance metric might have masked some range shifts that occurred  in sections of the coast that slant eastward. We avoided some of the most common sources of  methodological error in detecting range shifts using a long timeseries, focusing on abundant species, and calculating edge position  as a percentile rather than from the most extreme values (Bates  et al., 2015; Brown et al., 2016). However, the high degree of variability that we observed in species’ range edge dynamics may be  as much or more due to imperfect sampling than to actual fluctuations in the species range.

%These correlations suggest that, although cross-regional coherence is limited, within-region responses often align with local thermal trajectories, highlighting the multidimensional nature of marine species’ climate responses.



% our study is not global, yet it cover severeal region with different warming trends. one strength of our approach is that relies on standaridzed data and are modelled together. fuchs this may explain


% something about our study may be less extned in space that some more global analysis. Yet our study overcome bla bla we us e uncertainty








%realized shifts were generally larger in magnitude than tyexpected

%Surprisingly, we found 


%As a result, population-level range shifts are mediated by a combination of environmental, ecological, and physiological constraints, producing diverse redistribution pathways rather than uniform, directional movements.

%This weak cross-regional signal suggests that, although temperature declines with latitude and depth at broad scales, regional climate velocities form a complex mosaic that does not necessarily align range shifts along poleward or deepward axes . Collectively, these findings challenge the prevailing expectation that marine species consistently move poleward or deeper to maintain thermal affinity. Instead, they reveal a diversity of redistribution pathways and underscore the limits of assuming uniform, climate-driven tracking of cooler habitats. Such heterogeneity highlights that range shifts are context-dependent responses shaped by regional oceanography, species’ life histories, and ecological constraints.

%Final sentence


%This weak cross-regional signal may reflect that, while temperature generally declines with latitude and depth at broad scales, local climate velocities form a complex mosaic in which the directions of range shifts are not necessarily aligned poleward or deepward .
%Taken together, these results reinforce that poleward and deepward responses are neither as widespread nor as coherent as often presumed. Given this, we argue against the common expectation that species consistently move poleward or deeper to maintain thermal affinity. Instead, our findings highlight the diversity of redistribution strategies and the limits of simple assumptions that species universally track cooler habitats along predictable gradients. Such heterogeneity underscores the need to view range shifts not as uniform climate-driven migrations, but as context-dependent processes shaped by regional oceanography, life-history traits, and ecological constraints.




%Although some species and communities deepening, these responses were inconsistent overall. This pattern aligns with global syntheses showing similarly limited vertical movement. For instance, \citep{rubenstein_climate_2023} reported no systematic tendency for marine species to shift deeper in their meta-analysis of 315 studies spanning over 12,000 taxa. Such weak and variable depth responses contrast with the theoretical expectation that species should move downward to buffer against surface warming \citep{brown_effects_2015}. Depth-based redistribution therefore appears to be a secondary or constrained response, likely limited by habitat availability, oxygen gradients, and other ecological or physiological barriers.







%Taken together, these results reinforce that poleward and deepward responses are neither as widespread nor as coherent as often presumed. Given this, we argue against the common expectation that species consistently move poleward or deeper to maintain thermal affinity. Instead, our findings highlight the diversity of redistribution strategies and the limits of simple assumptions that species universally track cooler habitats along predictable gradients. Such heterogeneity underscores the need to view range shifts not as uniform climate-driven migrations, but as context-dependent processes shaped by regional oceanography, life-history traits, and ecological constraints.
%This overall lack of consistent poleward or deepward movement does not imply that all regions or species deviate from expectations. Some systems did exhibit coherent shifts consistent with thermal tracking. For example, in the Eastern Bering Sea, 64\% of species showed northward movement (compared to 21\% moving south), while in British Columbia 34\% shifted deeper and only 3\% shallower. Similarly, North Sea cod (\textit{Gadus morhua}) exhibited both northward and deepward shifts (xx and xxkm or mdecade$^{-1}$), a pattern increasingly well documented in the literature \citep{perry_climate_2005, engelhard_climate_2014}, tracking it's thermal niche (Fig x). 












%These complexities have important implications for predicting and managing biodiversity under climate change. Models that assume uniform poleward or deepward tracking are likely to overestimate species’ capacity to maintain suitable conditions. Incorporating multidimensional movement and exposure metrics, as done here, provides a more realistic view of how populations reorganize under warming. Such approaches can improve assessments of climate vulnerability and inform adaptive management strategies that account for spatial and thermal mismatches between species and their environments.

% this doens't mean species are not shifting north folowing expectations. for instance cod 

% tremendous variation
% severeal studies show methodological variation explain a lot of shift, here we.. 



% depth correlates with lat north sea 8. R. J. Knijn, T. W. Boon, H. J. L. Heessen, J. F. G. Hislop, Atlas of North Sea Fishes [International Council for the Exploration of the Sea (ICES) Cooperative Research Report 194, ICES, Copenhagen, 1993].

% Lenoir and svenning about multiple



%and highlights the importance of accounting for regional variability in climate trajectories, habitat geometry, and species’ ecological constraints. The absence of a dominant poleward fingerprint suggests that latitudinal movement alone is not a universal pathway of thermal niche tracking.

\section{Methods}\label{methods}

\subsection{Data sources}

\textbf{Fish observations.} We compiled fish biomass densities data (kg km\textsuperscript{-2}) from 
a large-scale collection of standardized, fishery-independent bottom-trawl surveys \citep{maureaud_fishglob_data_2024}. Each trawl deployment (haul) recorded the catch by species, with biomass standardized by the sampled (``swept'') area, alongside the haul’s location, depth, and time. We selected surveys with a consistent sampling protocol and at least 15 years of coverage, resulting in 17 surveys across 11 regions in the North Atlantic and Northeast Pacific (Fig.~\ref{fig:map}).
To ensure robust estimates of species’ spatial and temporal dynamics, we kept only taxa that (i) made up 99\% of the cumulative regional biomass, (ii) occurred in at least 15\% of hauls in that region, and (iii) were sampled in at least two hauls per year. These criteria yielded 226 fish populations (species–region combinations) across all regions.

\noindent
\textbf{Environmental data.} We obtained bottom temperature data from the Copernicus Global Ocean Physics Reanalysis \citep{european_union-copernicus_marine_service_global_2018}, which provides monthly estimates at 1/12$^{\circ}$ ($\approx$7 km) resolution. While in situ measurements exist for some surveys in certain years, we used this modeled product to obtain temporally averaged and spatially consistent time series. Bathymetric data were derived from the GEBCO 2023 dataset \citep{gebco_bathymetric_compilation_group_2023_gebco_2023_2023}, which provides global seafloor depth at 0.00417° ($\approx$400 m) resolution.



\subsection{Spatiotemporal modeling}\label{sec:Species spatiotemporal modeling}

\textbf{Model structure.} We modeled species distributions and their temporal dynamics using spatiotemporal generalized linear mixed-effects models (GLMMs), a flexible approach widely used in fisheries science \citep{thorson_geostatistical_2015, thorson_model-based_2016}. For each species (S1 Table), biomass density was modeled with a Tweedie distribution, using year (as a factor) and log-transformed depth (modeled as a second-order polynomial) as predictors. Survey identity and quarter were included as fixed effects to account for differences in sampling and seasonal variation when appropriate. Spatial and spatiotemporal variation was captured through random fields, approximated via the stochastic partial differential equation (SPDE) method with Gaussian Markov random fields \citep{lindgren_explicit_2011}. Models were implemented in R with the \texttt{sdmTMB} package \citep{anderson_sdmtmb_2024}, which integrates finite-element meshes constructed with \texttt{fmesher} \citep{lindgren_fmesher_2025} into Template Model Builder (TMB) \citep{kristensen_tmb_2016}. Full details on model structure, parameterization, and fitting procedures are provided in Appendix A.




\textbf{Prediction grid and environmental matching.} 
After model fitting and validation we predicted species-specific densities on a 4 × 4 km spatial
grid (in local UTM coordinates) covering each region for all years with available data. For each grid cell, we assigned bathymetry and the average bottom temperature over the 12 months preceding the earliest survey month in a region. This procedure generated a consistent spatiotemporal dataset of predicted biomass, depth, and temperature for downstream analyses.

\textbf{Derivation of multi-dimensional distributional metrics and abundance.} To assess species responses to ocean warming, we computed annual metrics that characterize both spatial redistribution and thermal exposure, specifically range centroids (UTM northing and easting), realized depth niche, and realized thermal niche. Horizontal shifts in distribution were quantified using range centroids rather than range edges, as centroids are less susceptible to noise and provide a clearer signal of overall distributional movement, thus offering a more stable and representative measure of geographic displacement \citep{shoo_detecting_2006}. Range centroids were calculated as the biomass-weighted mean UTM northing and easting (center of gravity) across grid cells. The realized depth and thermal niches were computed as biomass-weighted averages of depth and bottom temperature, respectively, summarizing the environmental conditions most intensely occupied by each species (Table~\ref{table:derived_quantities}). All these metrics were derived from biomass density predictions across the spatial grid generated by the fitted spatiotemporal models.


\begin{sidewaystable}[!htbp]
\caption{Summary of annual spatial and thermal metrics used to quantify species’ distributional and thermal responses to ocean warming.}
\label{table:derived_quantities}
\centering
\renewcommand{\arraystretch}{1.5}
\begin{tabular}{@{}p{3.2cm}p{3.5cm}p{5cm}p{4.3cm}@{}}
\toprule
Name & Metric & Definition / Formula & Interpretation \\
\midrule
Range centroid (UTM northing) & Density-weighted northing &
$\bar{y} = \frac{\sum y_i D_i}{\sum D_i}$, where $y_i$ is the UTM northing of grid cell $i$, and $D_i$ is predicted density. &
Tracks north–south distributional shifts. \\
Range centroid (UTM easting) & Density-weighted easting &
$\bar{x} = \frac{\sum x_i D_i}{\sum D_i}$, where $x_i$ is the UTM easting of grid cell $i$. &
Tracks east–west distributional shifts. \\
Depth niche & Density-weighted depth &
$\bar{z} = \frac{\sum z_i D_i}{\sum D_i}$, where $z_i$ is depth in grid cell $i$. &
Captures changes in vertical habitat use. \\
Thermal niche & Density-weighted temperature &
$\bar{T} = \frac{\sum T_i D_i}{\sum D_i}$, where $T_i$ is bottom temperature in grid cell $i$. &
Reflects thermal conditions in occupied habitat. \\
\botrule
\end{tabular}
\end{sidewaystable}

%To estimate the annual abundance of each fish population, we calculated an area-weighted relative abundance index by multiplying predicted biomass density by grid cell area, summing these values across all grid cells, and applying a standard bias correction \citep{thorson_geostatistical_2015, thorsonImplementingGeneric2016}.

\textbf{Comparing realized shifts with thermal envelopes shifts.}
To estimate population-specific shifts under climate-driven thermal change, we fitted species distribution models using temperature predictors. These models followed the structure described in \textit{\nameref{sec:Species spatiotemporal modeling}}, but excluded spatiotemporal random fields to focus on direct temperature effects. As predictors, we used (i) the average bottom temperature over the 12 months preceding the earliest survey month in a region, and (ii) the maximum bottom temperature of the previous year. The latter was included to reflect the hypothesis that temperature extremes, rather than means, are more likely to limit species’ ranges \citep{sunday_thermal_2019}. We then derived spatial metrics (range centroid and depth niche) as in \textit{\nameref{sec:Derivation of multi-dimensional distributional metrics and abundance.}}.These temperature-only models provide climate envelope-based expectations under thermal forcing.

\subsection{Bayesian trend analysis}\label{sec:Bayesian trend analysis}

We estimated temporal trends in spatial and thermal metrics using Bayesian mixed-effects models that jointly represent multiple, potentially correlated responses. Each response was modeled using a Student-t likelihood to provide robustness to extreme values \citep[e.g.,][]{anderson_black-swan_2017}, with uncertainty from the spatiotemporal modeling propagated via observation-specific standard errors. Correlations among responses were captured through a region-specific multivariate normal (MVN) covariance structure, allowing shared variation among spatial and thermal metrics to be estimated explicitly. Temporal trends were modeled hierarchically as varying slopes, with a global effect of time, region-level deviations, and species-specific deviations nested within regions, enabling inference across biological scales. To improve interpretability and facilitate model fitting, response variables and the time predictor were mean-centered within each species–region combination, such that their means equal zero. Under this parameterization, intercepts were omitted and all coefficients represent deviations from the species–region mean. Time was rescaled to decades, allowing slope parameters to be interpreted directly as rates of change per decade.

Thus the model can be written as:

\begin{gather}
y_{i}^{(j)} \sim \operatorname{Student\text{-}t}
\left(\nu^{(j)}, 
\mu_{i}^{(j)}, 
\sqrt{(\sigma^{(j)})^{2} + (s^{(j)}_{i})^{2}}
\right),
\qquad j=1,\dots,4, 
\\
\mu_{i}^{(j)} =
\beta^{(j)} \cdot \text{decade}_{i}
+ \beta_{\text{reg}[r[i]]}^{(j)} \cdot \text{decade}_{i}
+ \beta_{\text{reg:sp}[r[i],s[i]]}^{(j)} \cdot \text{decade}_{i}\label{eq:linearpred},
\\
\beta_{\text{reg}[r]}^{(j)} \sim 
\operatorname{N}\!\left(0,\sigma_{\text{reg}}^{(j)}\right),
\\
\begin{bmatrix}
\beta_{\text{reg:sp}[r,s]}^{(1)} \\
\beta_{\text{reg:sp}[r,s]}^{(2)} \\
\beta_{\text{reg:sp}[r,s]}^{(3)} \\
\beta_{\text{reg:sp}[r,s]}^{(4)}
\end{bmatrix}
\sim 
\operatorname{MVN}\!\left(
\mathbf{0},\,
\mathbf{S}_{r}\,\mathbf{R}_{r}\,\mathbf{S}_{r}
\right),
\qquad \text{for each species}~s~\text{in region}~r
\\
\mathbf{S}_{r} =
\begin{bmatrix}
\sigma^{(1)}_{\text{reg:sp},r} & 0 & 0 & 0 \\
0 & \sigma^{(2)}_{\text{reg:sp},r} & 0 & 0 \\
0 & 0 & \sigma^{(3)}_{\text{reg:sp},r} & 0 \\
0 & 0 & 0 & \sigma^{(4)}_{\text{reg:sp},r}
\end{bmatrix},
\\
\mathbf{R}_{r} =
\begin{bmatrix}
1 & \rho_{12} & \rho_{13} & \rho_{14} \\
\rho_{12} & 1 & \rho_{23} & \rho_{24} \\
\rho_{13} & \rho_{23} & 1 & \rho_{34} \\
\rho_{14} & \rho_{24} & \rho_{34} & 1
\end{bmatrix}.
\end{gather}


\vspace{1em}



\noindent Here, $y^{(j)}_i$ denotes the response variable $j$ for observation $i$, where $j$ is the latitudinal centroid,
longitudinal centroid, depth niche or thermal niche (Table \ref{table:derived_quantities}). 
The term $s^{(j)}_i$ is the known standard error associated with observation $i$, $\sigma^{(j)}$ is a response-specific residual scale parameter, and $\nu^{(j)}$ is the degrees of freedom.

The linear predictor $\mu^{(j)}_i$ (Eq.~\ref{eq:linearpred}) models temporal trends as varying slopes, with a global effect of time $\beta^{(j)}$, region-specific deviations $\beta^{(j)}_{\text{reg}[r]}$, and species-specific deviations nested within region $\beta^{(j)}_{\text{reg:sp}[r,s]}$. 

Species-level slopes deviation were modelled jointly across responses using a MVN distribution with region-specific covariance matrices $\boldsymbol{\Sigma_r} = \mathbf{S}_{r}\mathbf{R}_{r}\,\mathbf{S}_{r}$. This structure allows temporal trends in different response variables to be correlated among species within the same region, as controlled by $\mathbf{R}_{r}$, while allowing both the magnitude and structure of these correlations to vary among regions.

Priors were weakly informative and guided by published rates of range shifts, depth changes, and ocean warming (Appendix B).

We fit models in R using the \texttt{brms} package \citep{burkner_brms_2017}, which interfaces with Stan via \texttt{rstan} \citep{stan2024,stan_development_team_rstan_2024}. Each model ran with four Markov chain Monte Carlo (MCMC) chains of 4,000 iterations, discarding the first 2,000 as warm-up. The remaining 2,000 samples per chain (8,000 total post–warm-up draws) formed the posterior distribution. Convergence was confirmed by $\hat{R} < 1.01$, absence of divergent transitions, and effective sample sizes $> 400$ for all key parameters \citep{vehtari_rank-normalization_2021}, as detailed in Appendix B.


\bibliography{references}% common bib file
%% if required, the content of .bbl file can be included here once bbl is generated
%%\input sn-article.bbl

\backmatter

\bmhead{Supplementary information}
Supplementary Figs. 1–2 and Appendices A–B providing additional information on spatiotemporal models and Bayesian trend analysis.

\bmhead{Acknowledgements}
We thank all those who contributed to the collection of the survey data and the FISHGLOB consortium for providing public access to the dataset and for discussions held in Oldenburg. F.M. also thanks Christofer Griffith for early discussions that helped shape this work.

\bmhead{Data availability}
All code used to perform the analyses and generate the figures is available at \url{https://github.com/federico-maioli/species_redistribution_tracking}.

\bmhead{Funding}
M.L. was supported by a research grant from the Swedish Research Council Formas (grant no. 2022-01511 to Max Lindmark).

\clearpage

\linenumbers
\resetlinenumber

\setcounter{page}{1}
\setcounter{equation}{0}



\renewcommand{\thetable}{S\arabic{table}}
\setcounter{table}{0}
\renewcommand{\thefigure}{S\arabic{figure}}
\setcounter{figure}{0}
\addtocontents{toc}{\protect\setcounter{tocdepth}{0}}
\renewcommand{\theHfigure}{S\arabic{figure}}
\renewcommand{\theHtable}{S\arabic{table}}



\newgeometry{left=2cm,right=2cm,top=2.5cm,bottom=2.5cm}

\section*{Supplementary Information}


\subsection*{Supplementary figures}\label{sec:supp_figs}

\begin{figure}[htbr]
\centering
\includegraphics[width=1\textwidth]{output/figures/supp/posterior_slopes_supp}
    \caption{Species-specific trend shifts. Color shows median effect size; more intense colors indicate larger deviations from zero. Points and lines show median and 95\% CI. Marine regions: EBS = Eastern Bering Sea, GOA = Gulf of Alaska, BC = British Columbia, USWC = U.S. West Coast,  NEUS-SS = Northeast U.S. and Scotian Shelf, GOM = Gulf of Mexico, BS = Barents Sea, NS = North Sea, CBS = Celtic–Biscay Shelf,  BAL = Baltic Sea, NIC = Northern Iberian Coast.}
    \label{fig:posterior_slopes_supp}
\end{figure}


\begin{figure}[htbr]
\centering
\includegraphics[width=1\textwidth]{output/figures/supp/corr_shifts_supp.png}
\caption{Heatmap of posterior mean correlations ($\rho$) between trends in latitude, longitude, depth, and realized thermal niche across regions. Colors indicate the strength and direction of associations.}\label{fig:corr_shifts_supp}
\end{figure}



\begingroup\fontsize{6}{8}\selectfont

\begin{longtable}[t]{llllll}
\toprule
Region & Species & Latitudinal shift & Longitudinal shift & Depth shift & Thermal-niche shift\\
\midrule
\endfirsthead
\multicolumn{6}{@{}l}{\textit{(continued)}}\\
\toprule
Region & Species & Latitudinal shift & Longitudinal shift & Depth shift & Thermal-niche shift\\
\midrule
\endhead

\endfoot
\bottomrule
\endlastfoot
\cellcolor{gray!10}{EBS} & \cellcolor{gray!10}{Atheresthes\_evermanni} & \cellcolor{gray!10}{14.74 [5.61, 23.49]} & \cellcolor{gray!10}{14.54 [3.00, 25.68]} & \cellcolor{gray!10}{-4.70 [-6.03, -3.31]} & \cellcolor{gray!10}{-0.09 [-0.22, 0.03]}\\
EBS & Atheresthes\_stomias & 36.52 [28.85, 43.61] & -31.53 [-46.32, -15.51] & -2.71 [-4.19, -1.26] & -0.17 [-0.32, -0.03]\\
\cellcolor{gray!10}{EBS} & \cellcolor{gray!10}{Gadus\_chalcogrammus} & \cellcolor{gray!10}{35.25 [24.20, 44.31]} & \cellcolor{gray!10}{-24.49 [-43.91, -5.47]} & \cellcolor{gray!10}{0.78 [-1.45, 2.83]} & \cellcolor{gray!10}{-0.07 [-0.21, 0.08]}\\
EBS & Gadus\_macrocephalus & 16.48 [10.11, 22.53] & 1.00 [-11.54, 14.80] & -2.60 [-4.05, -1.22] & -0.07 [-0.20, 0.06]\\
\cellcolor{gray!10}{EBS} & \cellcolor{gray!10}{Hemilepidotus\_jordani} & \cellcolor{gray!10}{-12.43 [-18.12, -7.28]} & \cellcolor{gray!10}{6.04 [-2.49, 14.45]} & \cellcolor{gray!10}{-0.82 [-1.96, 0.25]} & \cellcolor{gray!10}{0.10 [-0.06, 0.27]}\\
EBS & Hemitripterus\_bolini & 43.10 [34.13, 51.79] & -45.97 [-59.35, -33.54] & 0.68 [-0.69, 2.07] & -0.14 [-0.30, 0.01]\\
\cellcolor{gray!10}{EBS} & \cellcolor{gray!10}{Hippoglossoides\_elassodon} & \cellcolor{gray!10}{11.75 [3.36, 18.28]} & \cellcolor{gray!10}{-17.78 [-28.62, -6.80]} & \cellcolor{gray!10}{-0.44 [-1.69, 0.83]} & \cellcolor{gray!10}{-0.00 [-0.12, 0.13]}\\
EBS & Hippoglossoides\_robustus & 15.33 [1.56, 23.86] & -16.90 [-26.35, -7.75] & 3.07 [1.19, 4.84] & 0.05 [-0.10, 0.21]\\
\cellcolor{gray!10}{EBS} & \cellcolor{gray!10}{Hippoglossus\_stenolepis} & \cellcolor{gray!10}{13.41 [8.30, 18.67]} & \cellcolor{gray!10}{27.88 [15.65, 43.85]} & \cellcolor{gray!10}{-7.16 [-8.63, -5.71]} & \cellcolor{gray!10}{-0.11 [-0.27, 0.05]}\\
EBS & Limanda\_aspera & -2.12 [-6.43, 1.96] & 12.61 [-0.57, 25.00] & 0.18 [-0.92, 1.19] & 0.07 [-0.08, 0.25]\\
\cellcolor{gray!10}{EBS} & \cellcolor{gray!10}{Platichthys\_stellatus} & \cellcolor{gray!10}{-9.25 [-15.86, -0.93]} & \cellcolor{gray!10}{-11.83 [-23.05, -0.98]} & \cellcolor{gray!10}{1.14 [-0.16, 2.35]} & \cellcolor{gray!10}{0.11 [-0.06, 0.28]}\\
EBS & Pleuronectes\_quadrituberculatus & -6.51 [-11.97, -1.33] & 9.01 [-2.21, 20.45] & 0.11 [-1.12, 1.34] & 0.07 [-0.09, 0.23]\\
\cellcolor{gray!10}{EBS} & \cellcolor{gray!10}{Podothecus\_accipenserinus} & \cellcolor{gray!10}{3.12 [-3.24, 10.47]} & \cellcolor{gray!10}{-23.38 [-37.89, -10.36]} & \cellcolor{gray!10}{1.18 [-0.10, 2.39]} & \cellcolor{gray!10}{0.04 [-0.12, 0.20]}\\
EBS & Reinhardtius\_hippoglossoides & 8.61 [1.13, 16.83] & 23.88 [14.89, 32.66] & -6.76 [-8.65, -4.66] & -0.11 [-0.26, 0.02]\\
\cellcolor{gray!10}{GOA} & \cellcolor{gray!10}{Anoplopoma\_fimbria} & \cellcolor{gray!10}{-10.21 [-28.11, 6.44]} & \cellcolor{gray!10}{-6.48 [-36.67, 22.10]} & \cellcolor{gray!10}{0.87 [-1.77, 6.73]} & \cellcolor{gray!10}{0.14 [-0.00, 0.24]}\\
GOA & Atheresthes\_stomias & 6.09 [-4.69, 16.81] & 5.79 [-11.61, 24.93] & -0.30 [-2.42, 1.16] & 0.16 [0.06, 0.27]\\
\cellcolor{gray!10}{GOA} & \cellcolor{gray!10}{Beringraja\_rhina} & \cellcolor{gray!10}{-1.17 [-12.82, 9.85]} & \cellcolor{gray!10}{-1.51 [-20.97, 17.27]} & \cellcolor{gray!10}{-0.08 [-2.35, 1.58]} & \cellcolor{gray!10}{0.15 [0.05, 0.25]}\\
GOA & Gadus\_chalcogrammus & -8.00 [-27.25, 9.13] & -7.15 [-40.28, 22.45] & 0.67 [-1.93, 4.82] & 0.16 [0.04, 0.28]\\
\cellcolor{gray!10}{GOA} & \cellcolor{gray!10}{Gadus\_macrocephalus} & \cellcolor{gray!10}{11.67 [-4.26, 27.88]} & \cellcolor{gray!10}{19.18 [-5.03, 48.03]} & \cellcolor{gray!10}{1.25 [-0.54, 3.87]} & \cellcolor{gray!10}{0.17 [0.06, 0.31]}\\
GOA & Glyptocephalus\_zachirus & -9.08 [-29.33, 6.69] & -20.95 [-52.48, 3.99] & 0.08 [-2.24, 2.20] & 0.15 [0.02, 0.26]\\
\cellcolor{gray!10}{GOA} & \cellcolor{gray!10}{Hemilepidotus\_jordani} & \cellcolor{gray!10}{3.40 [-7.99, 17.15]} & \cellcolor{gray!10}{5.56 [-17.10, 32.66]} & \cellcolor{gray!10}{0.45 [-0.98, 1.94]} & \cellcolor{gray!10}{0.16 [0.06, 0.28]}\\
GOA & Hippoglossoides\_elassodon & -0.43 [-19.01, 17.21] & -1.04 [-22.36, 19.40] & 0.35 [-1.46, 2.17] & 0.16 [0.05, 0.27]\\
\cellcolor{gray!10}{GOA} & \cellcolor{gray!10}{Hippoglossus\_stenolepis} & \cellcolor{gray!10}{5.05 [-3.75, 15.03]} & \cellcolor{gray!10}{12.27 [-4.20, 30.75]} & \cellcolor{gray!10}{0.94 [-0.38, 2.68]} & \cellcolor{gray!10}{0.16 [0.06, 0.29]}\\
GOA & Microstomus\_pacificus & 9.41 [-0.72, 20.60] & 20.29 [-2.16, 47.75] & 0.18 [-3.58, 2.86] & 0.16 [0.05, 0.27]\\
\cellcolor{gray!10}{GOA} & \cellcolor{gray!10}{Sebastes\_alutus} & \cellcolor{gray!10}{18.72 [-1.32, 44.92]} & \cellcolor{gray!10}{13.18 [-20.55, 55.89]} & \cellcolor{gray!10}{-1.04 [-7.78, 1.51]} & \cellcolor{gray!10}{0.16 [0.05, 0.29]}\\
GOA & Sebastes\_polyspinis & -8.21 [-25.86, 7.23] & -9.49 [-42.09, 17.20] & -0.43 [-4.55, 1.71] & 0.15 [0.03, 0.26]\\
\cellcolor{gray!10}{GOA} & \cellcolor{gray!10}{Sebastes\_variabilis} & \cellcolor{gray!10}{-2.10 [-23.35, 17.70]} & \cellcolor{gray!10}{-3.42 [-35.78, 24.07]} & \cellcolor{gray!10}{1.14 [-0.82, 4.73]} & \cellcolor{gray!10}{0.16 [0.05, 0.27]}\\
GOA & Sebastolobus\_alascanus & -1.82 [-19.78, 14.23] & -3.63 [-23.91, 14.95] & 0.36 [-3.25, 4.33] & 0.15 [0.04, 0.26]\\
\cellcolor{gray!10}{GOA} & \cellcolor{gray!10}{Squalus\_suckleyi} & \cellcolor{gray!10}{-2.98 [-15.45, 9.76]} & \cellcolor{gray!10}{-5.02 [-44.26, 23.54]} & \cellcolor{gray!10}{0.75 [-1.70, 5.65]} & \cellcolor{gray!10}{0.16 [0.05, 0.27]}\\
BC & Anoplopoma\_fimbria & -83.23 [-121.29, -46.41] & 66.71 [39.77, 95.53] & -15.84 [-37.22, 0.99] & 0.54 [0.33, 0.80]\\
\cellcolor{gray!10}{BC} & \cellcolor{gray!10}{Atheresthes\_stomias} & \cellcolor{gray!10}{-64.80 [-94.42, -35.29]} & \cellcolor{gray!10}{45.56 [25.19, 65.68]} & \cellcolor{gray!10}{0.60 [-7.89, 8.90]} & \cellcolor{gray!10}{0.36 [0.22, 0.52]}\\
BC & Beringraja\_rhina & -5.93 [-26.25, 15.42] & 0.53 [-15.19, 15.96] & 6.73 [-3.24, 16.96] & 0.25 [0.09, 0.38]\\
\cellcolor{gray!10}{BC} & \cellcolor{gray!10}{Citharichthys\_sordidus} & \cellcolor{gray!10}{-65.47 [-106.31, -25.41]} & \cellcolor{gray!10}{49.69 [20.74, 78.09]} & \cellcolor{gray!10}{-3.18 [-6.81, 0.21]} & \cellcolor{gray!10}{0.41 [0.27, 0.57]}\\
BC & Eopsetta\_jordani & -54.75 [-84.30, -28.85] & 38.11 [18.58, 60.18] & 4.93 [2.14, 7.69] & 0.31 [0.19, 0.45]\\
\cellcolor{gray!10}{BC} & \cellcolor{gray!10}{Gadus\_chalcogrammus} & \cellcolor{gray!10}{-43.28 [-68.47, -20.41]} & \cellcolor{gray!10}{27.51 [9.79, 46.46]} & \cellcolor{gray!10}{7.77 [0.62, 14.13]} & \cellcolor{gray!10}{0.29 [0.15, 0.43]}\\
BC & Gadus\_macrocephalus & -24.20 [-46.73, -2.26] & 12.63 [-4.16, 30.19] & 10.59 [7.22, 14.07] & 0.23 [0.08, 0.36]\\
\cellcolor{gray!10}{BC} & \cellcolor{gray!10}{Glyptocephalus\_zachirus} & \cellcolor{gray!10}{-79.32 [-103.69, -56.83]} & \cellcolor{gray!10}{57.53 [40.96, 75.09]} & \cellcolor{gray!10}{-3.11 [-6.77, 0.79]} & \cellcolor{gray!10}{0.41 [0.27, 0.57]}\\
BC & Hippoglossoides\_elassodon & -0.43 [-37.64, 35.93] & -2.80 [-30.16, 25.04] & 5.78 [0.48, 10.75] & 0.26 [0.12, 0.40]\\
\cellcolor{gray!10}{BC} & \cellcolor{gray!10}{Hippoglossus\_stenolepis} & \cellcolor{gray!10}{53.14 [33.22, 73.78]} & \cellcolor{gray!10}{-41.73 [-57.82, -26.14]} & \cellcolor{gray!10}{8.10 [4.34, 12.12]} & \cellcolor{gray!10}{0.18 [0.01, 0.33]}\\
BC & Hydrolagus\_colliei & 26.80 [1.74, 51.33] & -18.01 [-36.02, 0.79] & 1.92 [-2.01, 5.65] & 0.25 [0.10, 0.39]\\
\cellcolor{gray!10}{BC} & \cellcolor{gray!10}{Lepidopsetta\_bilineata} & \cellcolor{gray!10}{-3.64 [-25.09, 17.67]} & \cellcolor{gray!10}{2.26 [-14.20, 18.46]} & \cellcolor{gray!10}{1.01 [-1.45, 3.38]} & \cellcolor{gray!10}{0.29 [0.16, 0.42]}\\
BC & Lyopsetta\_exilis & -39.94 [-61.79, -19.30] & 30.42 [13.83, 47.19] & -3.49 [-7.19, 0.23] & 0.38 [0.25, 0.52]\\
\cellcolor{gray!10}{BC} & \cellcolor{gray!10}{Merluccius\_productus} & \cellcolor{gray!10}{-80.88 [-132.67, -32.22]} & \cellcolor{gray!10}{61.16 [25.26, 100.41]} & \cellcolor{gray!10}{-7.81 [-26.98, 9.61]} & \cellcolor{gray!10}{0.45 [0.26, 0.72]}\\
BC & Microstomus\_pacificus & -79.92 [-111.88, -49.39] & 54.04 [32.70, 75.89] & 6.90 [-8.04, 22.48] & 0.31 [0.12, 0.49]\\
\cellcolor{gray!10}{BC} & \cellcolor{gray!10}{Ophiodon\_elongatus} & \cellcolor{gray!10}{53.10 [33.38, 74.05]} & \cellcolor{gray!10}{-40.52 [-57.01, -24.55]} & \cellcolor{gray!10}{-0.09 [-2.85, 2.64]} & \cellcolor{gray!10}{0.26 [0.11, 0.42]}\\
BC & Parophrys\_vetulus & -33.46 [-63.08, -4.25] & 21.20 [0.42, 42.63] & 6.41 [3.22, 9.50] & 0.28 [0.14, 0.41]\\
\cellcolor{gray!10}{BC} & \cellcolor{gray!10}{Sebastes\_alutus} & \cellcolor{gray!10}{-8.98 [-38.79, 19.33]} & \cellcolor{gray!10}{2.70 [-19.87, 24.97]} & \cellcolor{gray!10}{6.77 [1.47, 12.27]} & \cellcolor{gray!10}{0.24 [0.09, 0.36]}\\
BC & Sebastes\_babcocki & -1.61 [-17.90, 14.58] & -5.03 [-18.88, 8.55] & 11.61 [3.31, 20.51] & 0.20 [0.04, 0.34]\\
\cellcolor{gray!10}{BC} & \cellcolor{gray!10}{Sebastes\_brevispinis} & \cellcolor{gray!10}{-9.69 [-36.38, 14.96]} & \cellcolor{gray!10}{4.13 [-16.29, 25.71]} & \cellcolor{gray!10}{3.59 [0.10, 7.04]} & \cellcolor{gray!10}{0.29 [0.16, 0.41]}\\
BC & Sebastes\_elongatus & 18.57 [-1.79, 40.07] & -14.82 [-31.62, 1.54] & 1.19 [-1.62, 3.97] & 0.28 [0.15, 0.40]\\
\cellcolor{gray!10}{BC} & \cellcolor{gray!10}{Sebastes\_flavidus} & \cellcolor{gray!10}{-77.82 [-122.57, -30.61]} & \cellcolor{gray!10}{54.88 [20.79, 85.82]} & \cellcolor{gray!10}{3.44 [-1.37, 8.26]} & \cellcolor{gray!10}{0.35 [0.20, 0.51]}\\
BC & Sebastes\_helvomaculatus & -7.11 [-40.38, 22.83] & 5.51 [-17.64, 29.34] & -2.06 [-8.15, 3.84] & 0.32 [0.18, 0.45]\\
\cellcolor{gray!10}{BC} & \cellcolor{gray!10}{Sebastes\_pinniger} & \cellcolor{gray!10}{47.48 [12.88, 84.13]} & \cellcolor{gray!10}{-32.38 [-60.11, -5.65]} & \cellcolor{gray!10}{-9.75 [-15.30, -4.19]} & \cellcolor{gray!10}{0.35 [0.19, 0.54]}\\
BC & Sebastes\_proriger & -2.65 [-35.46, 28.22] & 0.89 [-23.58, 26.05] & 2.26 [-6.14, 10.89] & 0.28 [0.14, 0.43]\\
\cellcolor{gray!10}{BC} & \cellcolor{gray!10}{Sebastes\_reedi} & \cellcolor{gray!10}{38.68 [21.09, 56.07]} & \cellcolor{gray!10}{-27.38 [-43.25, -11.65]} & \cellcolor{gray!10}{-1.93 [-14.16, 8.78]} & \cellcolor{gray!10}{0.28 [0.11, 0.46]}\\
BC & Sebastes\_zacentrus & -97.66 [-137.41, -50.58] & 72.97 [38.81, 103.91] & -7.71 [-17.02, 1.57] & 0.46 [0.28, 0.65]\\
\cellcolor{gray!10}{BC} & \cellcolor{gray!10}{Sebastolobus\_alascanus} & \cellcolor{gray!10}{-0.78 [-21.27, 19.94]} & \cellcolor{gray!10}{-2.33 [-18.61, 13.43]} & \cellcolor{gray!10}{4.49 [-5.79, 14.88]} & \cellcolor{gray!10}{0.25 [0.09, 0.38]}\\
BC & Squalus\_suckleyi & -13.33 [-55.57, 27.34] & -3.12 [-32.37, 28.58] & 32.43 [25.21, 39.49] & -0.01 [-0.27, 0.25]\\
\cellcolor{gray!10}{USWC} & \cellcolor{gray!10}{Albatrossia\_pectoralis} & \cellcolor{gray!10}{7.65 [-24.20, 37.80]} & \cellcolor{gray!10}{-1.90 [-13.42, 10.03]} & \cellcolor{gray!10}{1.37 [-2.27, 5.03]} & \cellcolor{gray!10}{0.11 [0.01, 0.19]}\\
USWC & Alepocephalus\_tenebrosus & 34.78 [11.79, 57.12] & -12.30 [-27.04, -0.32] & 0.30 [-3.82, 4.23] & 0.10 [-0.01, 0.19]\\
\cellcolor{gray!10}{USWC} & \cellcolor{gray!10}{Anoplopoma\_fimbria} & \cellcolor{gray!10}{73.11 [22.81, 120.14]} & \cellcolor{gray!10}{-16.91 [-34.32, -1.01]} & \cellcolor{gray!10}{-3.27 [-13.99, 5.34]} & \cellcolor{gray!10}{0.10 [-0.03, 0.21]}\\
USWC & Apristurus\_brunneus & -50.38 [-93.88, -5.55] & 13.39 [-2.51, 32.91] & 3.48 [-3.17, 10.07] & 0.12 [0.03, 0.24]\\
\cellcolor{gray!10}{USWC} & \cellcolor{gray!10}{Atheresthes\_stomias} & \cellcolor{gray!10}{-106.44 [-136.09, -77.89]} & \cellcolor{gray!10}{13.72 [1.72, 25.15]} & \cellcolor{gray!10}{12.58 [7.65, 17.03]} & \cellcolor{gray!10}{0.12 [0.01, 0.28]}\\
USWC & Beringraja\_binoculata & 33.57 [-13.89, 91.95] & -9.46 [-22.12, 3.39] & 1.95 [-0.44, 4.46] & 0.11 [0.01, 0.21]\\
\cellcolor{gray!10}{USWC} & \cellcolor{gray!10}{Beringraja\_rhina} & \cellcolor{gray!10}{-21.12 [-49.75, 7.83]} & \cellcolor{gray!10}{5.33 [-5.79, 17.08]} & \cellcolor{gray!10}{3.82 [-0.21, 7.92]} & \cellcolor{gray!10}{0.12 [0.03, 0.22]}\\
USWC & Bothrocara\_brunneum & -19.53 [-80.93, 37.14] & 5.14 [-10.68, 22.84] & 0.31 [-7.34, 7.06] & 0.11 [0.01, 0.22]\\
\cellcolor{gray!10}{USWC} & \cellcolor{gray!10}{Citharichthys\_sordidus} & \cellcolor{gray!10}{-25.11 [-91.07, 40.65]} & \cellcolor{gray!10}{5.60 [-15.27, 29.26]} & \cellcolor{gray!10}{2.92 [0.40, 5.65]} & \cellcolor{gray!10}{0.12 [0.03, 0.24]}\\
USWC & Coryphaenoides\_acrolepis & 23.83 [-10.83, 60.53] & -7.02 [-20.49, 5.88] & 0.41 [-2.58, 3.25] & 0.10 [-0.00, 0.19]\\
\cellcolor{gray!10}{USWC} & \cellcolor{gray!10}{Eopsetta\_jordani} & \cellcolor{gray!10}{50.77 [12.24, 94.36]} & \cellcolor{gray!10}{-12.33 [-26.36, 0.71]} & \cellcolor{gray!10}{2.42 [-0.08, 5.01]} & \cellcolor{gray!10}{0.10 [-0.01, 0.20]}\\
USWC & Glyptocephalus\_zachirus & 70.91 [20.29, 122.45] & -12.32 [-29.83, 5.00] & -6.45 [-13.46, -0.19] & 0.11 [-0.01, 0.22]\\
\cellcolor{gray!10}{USWC} & \cellcolor{gray!10}{Hydrolagus\_colliei} & \cellcolor{gray!10}{0.96 [-51.27, 56.14]} & \cellcolor{gray!10}{-2.31 [-19.12, 13.01]} & \cellcolor{gray!10}{2.41 [-0.78, 5.47]} & \cellcolor{gray!10}{0.11 [0.02, 0.22]}\\
USWC & Lycodes\_cortezianus & 77.53 [30.66, 126.85] & -18.75 [-35.97, -3.73] & -2.67 [-7.73, 2.42] & 0.10 [-0.04, 0.19]\\
\cellcolor{gray!10}{USWC} & \cellcolor{gray!10}{Lycodes\_diapterus} & \cellcolor{gray!10}{-77.78 [-176.20, 13.51]} & \cellcolor{gray!10}{24.95 [-2.70, 68.48]} & \cellcolor{gray!10}{-0.75 [-14.92, 9.16]} & \cellcolor{gray!10}{0.14 [0.03, 0.38]}\\
USWC & Lycodes\_pacificus & -28.87 [-104.19, 44.12] & 4.96 [-17.24, 29.27] & 5.10 [-2.59, 13.73] & 0.12 [0.01, 0.24]\\
\cellcolor{gray!10}{USWC} & \cellcolor{gray!10}{Lyopsetta\_exilis} & \cellcolor{gray!10}{-22.72 [-70.33, 22.44]} & \cellcolor{gray!10}{10.60 [-2.86, 26.81]} & \cellcolor{gray!10}{1.97 [-1.60, 5.67]} & \cellcolor{gray!10}{0.12 [0.03, 0.24]}\\
USWC & Merluccius\_productus & -64.94 [-170.30, 9.64] & 17.50 [-2.88, 45.87] & 4.92 [-3.34, 14.89] & 0.12 [0.02, 0.26]\\
\cellcolor{gray!10}{USWC} & \cellcolor{gray!10}{Microstomus\_bathybius} & \cellcolor{gray!10}{1.17 [-12.71, 15.01]} & \cellcolor{gray!10}{-0.25 [-9.73, 9.65]} & \cellcolor{gray!10}{0.30 [-2.49, 3.05]} & \cellcolor{gray!10}{0.11 [0.02, 0.20]}\\
USWC & Microstomus\_pacificus & -12.75 [-45.56, 20.88] & 1.27 [-10.45, 13.13] & 5.67 [-0.64, 13.36] & 0.11 [0.00, 0.21]\\
\cellcolor{gray!10}{USWC} & \cellcolor{gray!10}{Ophiodon\_elongatus} & \cellcolor{gray!10}{-12.06 [-74.85, 47.60]} & \cellcolor{gray!10}{5.80 [-11.47, 26.78]} & \cellcolor{gray!10}{1.32 [-1.56, 4.13]} & \cellcolor{gray!10}{0.12 [0.03, 0.23]}\\
USWC & Parophrys\_vetulus & -27.33 [-66.24, 11.07] & 8.56 [-3.88, 23.83] & 3.53 [1.02, 6.18] & 0.12 [0.04, 0.24]\\
\cellcolor{gray!10}{USWC} & \cellcolor{gray!10}{Sebastes\_crameri} & \cellcolor{gray!10}{-38.18 [-85.25, 12.57]} & \cellcolor{gray!10}{0.59 [-11.17, 11.90]} & \cellcolor{gray!10}{9.05 [2.92, 16.36]} & \cellcolor{gray!10}{0.11 [0.00, 0.23]}\\
USWC & Sebastes\_diploproa & -4.81 [-58.86, 56.66] & -5.56 [-32.07, 14.38] & 5.94 [1.32, 10.58] & 0.11 [-0.00, 0.21]\\
\cellcolor{gray!10}{USWC} & \cellcolor{gray!10}{Sebastes\_elongatus} & \cellcolor{gray!10}{-21.54 [-84.33, 39.39]} & \cellcolor{gray!10}{4.40 [-14.61, 25.08]} & \cellcolor{gray!10}{2.94 [0.13, 5.75]} & \cellcolor{gray!10}{0.12 [0.02, 0.22]}\\
USWC & Sebastes\_saxicola & -35.26 [-57.65, -4.69] & 12.42 [-6.30, 40.05] & -0.46 [-9.40, 7.06] & 0.12 [0.01, 0.25]\\
\cellcolor{gray!10}{USWC} & \cellcolor{gray!10}{Sebastolobus\_alascanus} & \cellcolor{gray!10}{72.42 [36.24, 109.52]} & \cellcolor{gray!10}{-20.97 [-37.17, -6.61]} & \cellcolor{gray!10}{-4.92 [-11.83, 1.21]} & \cellcolor{gray!10}{0.10 [-0.04, 0.20]}\\
USWC & Sebastolobus\_altivelis & 4.40 [-10.39, 19.23] & -1.63 [-11.87, 8.60] & 0.58 [-1.85, 3.04] & 0.11 [0.02, 0.20]\\
\cellcolor{gray!10}{USWC} & \cellcolor{gray!10}{Squalus\_suckleyi} & \cellcolor{gray!10}{60.55 [-17.47, 164.14]} & \cellcolor{gray!10}{-18.43 [-39.11, 1.70]} & \cellcolor{gray!10}{3.74 [-1.32, 9.29]} & \cellcolor{gray!10}{0.10 [-0.05, 0.20]}\\
NEUS-SS & Alosa\_pseudoharengus & 3.33 [-14.24, 20.60] & -7.72 [-32.49, 14.59] & 6.61 [2.87, 10.69] & 0.54 [0.38, 0.70]\\
\cellcolor{gray!10}{NEUS-SS} & \cellcolor{gray!10}{Gadus\_morhua} & \cellcolor{gray!10}{7.85 [-11.38, 27.19]} & \cellcolor{gray!10}{19.46 [-12.71, 53.29]} & \cellcolor{gray!10}{-0.12 [-3.20, 2.85]} & \cellcolor{gray!10}{0.49 [0.32, 0.67]}\\
NEUS-SS & Glyptocephalus\_cynoglossus & 26.08 [14.57, 37.55] & 49.60 [29.27, 70.79] & -0.58 [-2.36, 1.23] & 0.37 [0.22, 0.53]\\
\cellcolor{gray!10}{NEUS-SS} & \cellcolor{gray!10}{Hemitripterus\_americanus} & \cellcolor{gray!10}{35.55 [24.22, 46.50]} & \cellcolor{gray!10}{36.20 [21.62, 51.24]} & \cellcolor{gray!10}{-0.79 [-1.97, 0.42]} & \cellcolor{gray!10}{0.34 [0.19, 0.49]}\\
NEUS-SS & Hippoglossina\_oblonga & 1.86 [-2.69, 6.73] & 2.71 [-8.90, 14.65] & 1.96 [-0.40, 4.34] & 0.53 [0.37, 0.67]\\
\cellcolor{gray!10}{NEUS-SS} & \cellcolor{gray!10}{Hippoglossoides\_platessoides} & \cellcolor{gray!10}{-12.81 [-24.85, -0.92]} & \cellcolor{gray!10}{-25.86 [-48.64, -2.11]} & \cellcolor{gray!10}{1.11 [-0.52, 2.88]} & \cellcolor{gray!10}{0.68 [0.54, 0.84]}\\
NEUS-SS & Leucoraja\_erinaceus & -28.40 [-37.85, -18.38] & -39.23 [-54.64, -23.80] & 0.44 [-0.62, 1.50] & 0.67 [0.45, 0.82]\\
\cellcolor{gray!10}{NEUS-SS} & \cellcolor{gray!10}{Leucoraja\_ocellata} & \cellcolor{gray!10}{-56.77 [-66.95, -47.48]} & \cellcolor{gray!10}{-73.48 [-94.59, -54.04]} & \cellcolor{gray!10}{-0.47 [-3.14, 2.45]} & \cellcolor{gray!10}{0.95 [0.78, 1.15]}\\
NEUS-SS & Lophius\_americanus & -5.56 [-15.46, 4.09] & -14.03 [-31.48, 2.99] & 2.20 [-0.75, 5.01] & 0.64 [0.51, 0.80]\\
\cellcolor{gray!10}{NEUS-SS} & \cellcolor{gray!10}{Melanogrammus\_aeglefinus} & \cellcolor{gray!10}{-55.89 [-65.88, -45.06]} & \cellcolor{gray!10}{-101.64 [-119.33, -82.26]} & \cellcolor{gray!10}{4.00 [1.36, 6.82]} & \cellcolor{gray!10}{0.97 [0.81, 1.15]}\\
NEUS-SS & Merluccius\_bilinearis & -32.28 [-41.20, -23.73] & -100.05 [-121.71, -78.02] & 6.29 [1.69, 10.61] & 0.84 [0.67, 1.01]\\
\cellcolor{gray!10}{NEUS-SS} & \cellcolor{gray!10}{Myoxocephalus\_octodecemspinosus} & \cellcolor{gray!10}{27.78 [14.61, 40.52]} & \cellcolor{gray!10}{23.88 [2.07, 45.86]} & \cellcolor{gray!10}{-0.84 [-2.37, 0.64]} & \cellcolor{gray!10}{0.39 [0.23, 0.55]}\\
NEUS-SS & Myzopsetta\_ferruginea & 67.06 [46.90, 86.15] & 98.42 [61.01, 133.76] & -1.92 [-3.49, -0.34] & 0.08 [-0.13, 0.27]\\
\cellcolor{gray!10}{NEUS-SS} & \cellcolor{gray!10}{Paralichthys\_dentatus} & \cellcolor{gray!10}{20.79 [10.37, 31.83]} & \cellcolor{gray!10}{29.11 [14.37, 43.53]} & \cellcolor{gray!10}{4.73 [2.57, 6.82]} & \cellcolor{gray!10}{0.37 [0.19, 0.53]}\\
NEUS-SS & Peprilus\_triacanthus & -25.61 [-45.92, -6.78] & -18.67 [-44.24, 5.12] & 1.44 [-2.28, 5.46] & 0.68 [0.48, 0.87]\\
\cellcolor{gray!10}{NEUS-SS} & \cellcolor{gray!10}{Pollachius\_virens} & \cellcolor{gray!10}{-7.78 [-15.07, -0.29]} & \cellcolor{gray!10}{-16.60 [-35.98, 0.94]} & \cellcolor{gray!10}{4.35 [1.08, 7.90]} & \cellcolor{gray!10}{0.64 [0.49, 0.79]}\\
NEUS-SS & Pseudopleuronectes\_americanus & 17.52 [8.75, 27.09] & 11.55 [1.09, 22.64] & 1.39 [0.07, 2.69] & 0.45 [0.31, 0.57]\\
\cellcolor{gray!10}{NEUS-SS} & \cellcolor{gray!10}{Scophthalmus\_aquosus} & \cellcolor{gray!10}{-6.43 [-13.16, 0.99]} & \cellcolor{gray!10}{-36.92 [-49.89, -24.21]} & \cellcolor{gray!10}{-1.83 [-3.48, -0.15]} & \cellcolor{gray!10}{0.54 [0.33, 0.72]}\\
NEUS-SS & Squalus\_acanthias & -83.30 [-105.77, -59.97] & -109.31 [-139.83, -78.31] & 2.39 [-2.30, 6.97] & 1.10 [0.91, 1.31]\\
\cellcolor{gray!10}{NEUS-SS} & \cellcolor{gray!10}{Urophycis\_chuss} & \cellcolor{gray!10}{-7.80 [-17.44, 1.29]} & \cellcolor{gray!10}{-30.23 [-44.46, -16.34]} & \cellcolor{gray!10}{3.55 [1.24, 5.90]} & \cellcolor{gray!10}{0.68 [0.55, 0.83]}\\
NEUS-SS & Urophycis\_regia & 1.23 [-9.69, 12.83] & 14.84 [2.12, 27.83] & 4.06 [0.38, 8.24] & 0.52 [0.33, 0.71]\\
\cellcolor{gray!10}{NEUS-SS} & \cellcolor{gray!10}{Urophycis\_tenuis} & \cellcolor{gray!10}{-20.90 [-28.63, -12.69]} & \cellcolor{gray!10}{-40.32 [-55.57, -24.98]} & \cellcolor{gray!10}{2.19 [0.01, 4.40]} & \cellcolor{gray!10}{0.75 [0.62, 0.91]}\\
NEUS-SS & Zoarces\_americanus & 8.47 [-9.00, 25.10] & 7.08 [-13.07, 28.20] & 1.45 [-1.17, 4.15] & 0.48 [0.33, 0.63]\\
\cellcolor{gray!10}{GOM} & \cellcolor{gray!10}{Balistes\_capriscus} & \cellcolor{gray!10}{-0.40 [-5.39, 4.73]} & \cellcolor{gray!10}{-8.53 [-25.34, 9.22]} & \cellcolor{gray!10}{-0.53 [-1.96, 0.87]} & \cellcolor{gray!10}{0.26 [0.16, 0.36]}\\
GOM & Chaetodipterus\_faber & -1.68 [-6.37, 3.10] & -23.91 [-41.66, -6.17] & -1.55 [-3.16, -0.14] & 0.30 [0.19, 0.40]\\
\cellcolor{gray!10}{GOM} & \cellcolor{gray!10}{Diplectrum\_bivittatum} & \cellcolor{gray!10}{5.87 [0.95, 11.32]} & \cellcolor{gray!10}{13.38 [-1.04, 27.87]} & \cellcolor{gray!10}{-0.46 [-1.42, 0.47]} & \cellcolor{gray!10}{0.24 [0.14, 0.34]}\\
GOM & Eucinostomus\_gula & -8.00 [-18.22, 2.22] & -26.31 [-65.52, 13.66] & 0.40 [-0.92, 1.79] & 0.22 [0.11, 0.32]\\
\cellcolor{gray!10}{GOM} & \cellcolor{gray!10}{Harengula\_jaguana} & \cellcolor{gray!10}{30.01 [21.40, 39.03]} & \cellcolor{gray!10}{116.22 [89.09, 142.86]} & \cellcolor{gray!10}{1.86 [0.51, 3.20]} & \cellcolor{gray!10}{0.15 [0.01, 0.28]}\\
GOM & Lagocephalus\_laevigatus & -1.89 [-6.65, 2.77] & -5.13 [-17.90, 8.35] & 0.34 [-1.03, 1.68] & 0.22 [0.12, 0.32]\\
\cellcolor{gray!10}{GOM} & \cellcolor{gray!10}{Lagodon\_rhomboides} & \cellcolor{gray!10}{2.88 [-2.99, 9.27]} & \cellcolor{gray!10}{4.25 [-18.18, 29.42]} & \cellcolor{gray!10}{-0.70 [-2.21, 0.81]} & \cellcolor{gray!10}{0.28 [0.18, 0.40]}\\
GOM & Larimus\_fasciatus & 8.96 [2.99, 15.22] & 31.90 [7.33, 57.32] & 0.60 [-0.60, 1.83] & 0.20 [0.09, 0.30]\\
\cellcolor{gray!10}{GOM} & \cellcolor{gray!10}{Leiostomus\_xanthurus} & \cellcolor{gray!10}{-3.59 [-9.65, 2.88]} & \cellcolor{gray!10}{-39.10 [-61.73, -15.91]} & \cellcolor{gray!10}{-3.27 [-5.26, -1.39]} & \cellcolor{gray!10}{0.39 [0.27, 0.54]}\\
GOM & Lepophidium\_brevibarbe & -9.09 [-14.24, -3.86] & -45.58 [-64.35, -26.40] & -0.92 [-2.29, 0.45] & 0.28 [0.17, 0.38]\\
\cellcolor{gray!10}{GOM} & \cellcolor{gray!10}{Lutjanus\_campechanus} & \cellcolor{gray!10}{-5.17 [-11.33, 0.64]} & \cellcolor{gray!10}{-6.10 [-24.66, 14.35]} & \cellcolor{gray!10}{2.41 [0.71, 4.32]} & \cellcolor{gray!10}{0.12 [-0.02, 0.24]}\\
GOM & Lutjanus\_synagris & 2.22 [-1.98, 6.41] & -2.52 [-18.73, 13.06] & -0.99 [-2.15, 0.09] & 0.29 [0.20, 0.39]\\
\cellcolor{gray!10}{GOM} & \cellcolor{gray!10}{Micropogonias\_undulatus} & \cellcolor{gray!10}{-2.00 [-6.51, 2.46]} & \cellcolor{gray!10}{-29.47 [-48.74, -10.31]} & \cellcolor{gray!10}{-1.53 [-3.03, -0.06]} & \cellcolor{gray!10}{0.28 [0.17, 0.39]}\\
GOM & Opisthonema\_oglinum & 0.20 [-5.38, 6.00] & -1.72 [-21.25, 17.41] & 0.11 [-0.70, 0.96] & 0.22 [0.11, 0.31]\\
\cellcolor{gray!10}{GOM} & \cellcolor{gray!10}{Peprilus\_burti} & \cellcolor{gray!10}{-4.68 [-9.30, -0.30]} & \cellcolor{gray!10}{-15.12 [-29.38, -1.33]} & \cellcolor{gray!10}{0.52 [-0.97, 2.04]} & \cellcolor{gray!10}{0.22 [0.11, 0.32]}\\
GOM & Porichthys\_plectrodon & 1.23 [-3.70, 5.84] & 3.88 [-13.39, 19.81] & 0.56 [-0.98, 2.18] & 0.20 [0.10, 0.30]\\
\cellcolor{gray!10}{GOM} & \cellcolor{gray!10}{Prionotus\_longispinosus} & \cellcolor{gray!10}{-12.89 [-17.60, -8.53]} & \cellcolor{gray!10}{-57.72 [-74.85, -41.63]} & \cellcolor{gray!10}{-0.53 [-2.01, 1.10]} & \cellcolor{gray!10}{0.25 [0.13, 0.36]}\\
GOM & Prionotus\_paralatus & -2.21 [-7.70, 2.90] & -25.35 [-45.81, -7.14] & -1.80 [-3.59, -0.11] & 0.34 [0.23, 0.48]\\
\cellcolor{gray!10}{GOM} & \cellcolor{gray!10}{Prionotus\_stearnsi} & \cellcolor{gray!10}{-2.47 [-9.31, 4.21]} & \cellcolor{gray!10}{-6.66 [-27.89, 13.70]} & \cellcolor{gray!10}{1.15 [-0.62, 3.18]} & \cellcolor{gray!10}{0.18 [0.06, 0.29]}\\
GOM & Pristipomoides\_aquilonaris & 3.28 [-0.54, 7.00] & 6.11 [-8.49, 20.89] & -0.31 [-1.64, 0.96] & 0.27 [0.18, 0.39]\\
\cellcolor{gray!10}{GOM} & \cellcolor{gray!10}{Saurida\_brasiliensis} & \cellcolor{gray!10}{2.50 [-2.63, 7.65]} & \cellcolor{gray!10}{7.18 [-8.67, 22.95]} & \cellcolor{gray!10}{0.76 [-0.54, 2.15]} & \cellcolor{gray!10}{0.19 [0.08, 0.29]}\\
GOM & Serranus\_atrobranchus & -9.28 [-16.03, -2.80] & -55.22 [-76.09, -34.86] & -2.76 [-4.78, -0.92] & 0.38 [0.26, 0.53]\\
\cellcolor{gray!10}{GOM} & \cellcolor{gray!10}{Sphoeroides\_parvus} & \cellcolor{gray!10}{-5.30 [-11.05, 0.47]} & \cellcolor{gray!10}{-18.18 [-35.92, -1.86]} & \cellcolor{gray!10}{0.53 [-0.47, 1.54]} & \cellcolor{gray!10}{0.20 [0.09, 0.29]}\\
GOM & Stenotomus\_caprinus & -2.65 [-7.24, 2.08] & -18.91 [-33.26, -2.02] & -0.95 [-2.29, 0.45] & 0.29 [0.19, 0.40]\\
\cellcolor{gray!10}{GOM} & \cellcolor{gray!10}{Synodus\_foetens} & \cellcolor{gray!10}{-0.86 [-4.21, 2.54]} & \cellcolor{gray!10}{-18.19 [-28.74, -7.98]} & \cellcolor{gray!10}{-1.80 [-2.87, -0.74]} & \cellcolor{gray!10}{0.32 [0.23, 0.42]}\\
GOM & Trichiurus\_lepturus & 0.12 [-4.38, 4.70] & -8.68 [-28.68, 11.79] & -0.01 [-1.51, 1.51] & 0.22 [0.11, 0.32]\\
\cellcolor{gray!10}{GOM} & \cellcolor{gray!10}{Trichopsetta\_ventralis} & \cellcolor{gray!10}{-12.54 [-18.40, -7.07]} & \cellcolor{gray!10}{-60.16 [-80.53, -39.97]} & \cellcolor{gray!10}{-1.92 [-3.38, -0.54]} & \cellcolor{gray!10}{0.35 [0.24, 0.48]}\\
GOM & Upeneus\_parvus & 2.14 [-2.70, 6.99] & 8.09 [-6.88, 22.59] & 0.72 [-0.66, 2.19] & 0.21 [0.10, 0.31]\\
\cellcolor{gray!10}{BS} & \cellcolor{gray!10}{Amblyraja\_radiata} & \cellcolor{gray!10}{61.93 [36.17, 89.04]} & \cellcolor{gray!10}{-20.52 [-42.03, 1.76]} & \cellcolor{gray!10}{-5.55 [-12.04, 0.73]} & \cellcolor{gray!10}{-0.11 [-0.33, 0.08]}\\
BS & Anarhichas\_denticulatus & 0.04 [-9.53, 9.23] & -0.20 [-15.54, 15.22] & -0.17 [-4.76, 4.60] & -0.01 [-0.19, 0.17]\\
\cellcolor{gray!10}{BS} & \cellcolor{gray!10}{Anarhichas\_lupus} & \cellcolor{gray!10}{5.70 [-42.02, 55.98]} & \cellcolor{gray!10}{5.04 [-15.14, 24.93]} & \cellcolor{gray!10}{-1.27 [-7.81, 5.16]} & \cellcolor{gray!10}{-0.04 [-0.26, 0.17]}\\
BS & Anarhichas\_minor & 13.81 [-15.65, 43.80] & -12.42 [-51.71, 26.18] & -0.96 [-8.78, 6.76] & -0.01 [-0.22, 0.21]\\
\cellcolor{gray!10}{BS} & \cellcolor{gray!10}{Artediellus\_atlanticus} & \cellcolor{gray!10}{26.57 [-10.44, 54.73]} & \cellcolor{gray!10}{3.10 [-20.52, 27.52]} & \cellcolor{gray!10}{-3.79 [-7.04, -0.52]} & \cellcolor{gray!10}{-0.10 [-0.37, 0.07]}\\
BS & Boreogadus\_saida & 40.66 [6.58, 71.78] & -16.47 [-59.88, 28.03] & -4.61 [-14.93, 4.25] & -0.11 [-0.39, 0.08]\\
\cellcolor{gray!10}{BS} & \cellcolor{gray!10}{Gadus\_morhua} & \cellcolor{gray!10}{-22.89 [-55.44, 11.32]} & \cellcolor{gray!10}{-20.78 [-51.36, 9.10]} & \cellcolor{gray!10}{5.09 [-1.59, 12.31]} & \cellcolor{gray!10}{0.07 [-0.13, 0.32]}\\
BS & Hippoglossoides\_platessoides & 13.67 [-2.80, 29.91] & 4.92 [-13.64, 22.51] & 1.21 [-2.45, 4.78] & -0.07 [-0.27, 0.09]\\
\cellcolor{gray!10}{BS} & \cellcolor{gray!10}{Leptoclinus\_maculatus} & \cellcolor{gray!10}{72.13 [33.15, 112.33]} & \cellcolor{gray!10}{-4.77 [-33.49, 23.13]} & \cellcolor{gray!10}{-0.18 [-3.87, 3.61]} & \cellcolor{gray!10}{-0.17 [-0.40, 0.02]}\\
BS & Melanogrammus\_aeglefinus & 4.03 [-54.37, 54.96] & 5.09 [-22.97, 36.18] & 0.06 [-8.76, 9.09] & 0.01 [-0.22, 0.33]\\
\cellcolor{gray!10}{BS} & \cellcolor{gray!10}{Micromesistius\_poutassou} & \cellcolor{gray!10}{21.68 [-1.77, 45.72]} & \cellcolor{gray!10}{-53.97 [-85.35, -24.02]} & \cellcolor{gray!10}{0.56 [-8.75, 10.24]} & \cellcolor{gray!10}{0.03 [-0.15, 0.26]}\\
BS & Reinhardtius\_hippoglossoides & -34.95 [-65.63, 0.41] & -3.66 [-29.13, 21.33] & 1.55 [-6.53, 10.26] & 0.02 [-0.18, 0.23]\\
\cellcolor{gray!10}{BS} & \cellcolor{gray!10}{Sebastes\_mentella} & \cellcolor{gray!10}{-16.02 [-38.78, 8.82]} & \cellcolor{gray!10}{84.44 [51.89, 110.22]} & \cellcolor{gray!10}{-10.59 [-16.63, -3.94]} & \cellcolor{gray!10}{-0.03 [-0.27, 0.23]}\\
BS & Sebastes\_norvegicus & 1.45 [-31.30, 33.45] & -9.94 [-44.55, 26.19] & -2.78 [-9.71, 3.58] & 0.02 [-0.17, 0.26]\\
\cellcolor{gray!10}{BS} & \cellcolor{gray!10}{Triglops\_nybelini} & \cellcolor{gray!10}{77.56 [34.42, 123.97]} & \cellcolor{gray!10}{-31.45 [-65.43, 2.36]} & \cellcolor{gray!10}{3.27 [-5.48, 13.10]} & \cellcolor{gray!10}{-0.16 [-0.42, 0.05]}\\
BS & Trisopterus\_esmarkii & -16.04 [-36.27, 1.44] & -61.78 [-85.72, -34.33] & -8.49 [-15.98, -1.67] & 0.22 [-0.05, 0.53]\\
\cellcolor{gray!10}{NS} & \cellcolor{gray!10}{Amblyraja\_radiata} & \cellcolor{gray!10}{5.32 [-5.12, 14.14]} & \cellcolor{gray!10}{3.95 [-5.34, 12.75]} & \cellcolor{gray!10}{0.79 [-0.86, 2.44]} & \cellcolor{gray!10}{0.26 [0.16, 0.38]}\\
NS & Eutrigla\_gurnardus & 24.43 [16.47, 32.07] & 19.94 [11.84, 27.73] & 1.03 [-0.47, 2.54] & 0.18 [0.03, 0.30]\\
\cellcolor{gray!10}{NS} & \cellcolor{gray!10}{Gadus\_morhua} & \cellcolor{gray!10}{95.07 [83.64, 106.70]} & \cellcolor{gray!10}{-66.21 [-80.93, -51.62]} & \cellcolor{gray!10}{11.87 [10.31, 13.41]} & \cellcolor{gray!10}{0.05 [-0.08, 0.19]}\\
NS & Hippoglossoides\_platessoides & 15.75 [7.15, 23.96] & -36.01 [-46.99, -25.37] & 3.65 [2.08, 5.13] & 0.26 [0.15, 0.38]\\
\cellcolor{gray!10}{NS} & \cellcolor{gray!10}{Limanda\_limanda} & \cellcolor{gray!10}{7.64 [1.47, 13.48]} & \cellcolor{gray!10}{-6.60 [-13.81, 0.45]} & \cellcolor{gray!10}{1.17 [0.32, 2.04]} & \cellcolor{gray!10}{0.24 [0.14, 0.34]}\\
NS & Lophius\_piscatorius & 8.14 [-3.23, 19.27] & -5.15 [-14.16, 3.50] & 0.74 [-0.63, 2.10] & 0.25 [0.15, 0.35]\\
\cellcolor{gray!10}{NS} & \cellcolor{gray!10}{Melanogrammus\_aeglefinus} & \cellcolor{gray!10}{-9.54 [-18.84, 0.66]} & \cellcolor{gray!10}{-26.48 [-35.45, -17.86]} & \cellcolor{gray!10}{-3.34 [-4.60, -2.06]} & \cellcolor{gray!10}{0.34 [0.21, 0.47]}\\
NS & Merlangius\_merlangus & -46.09 [-59.37, -32.89] & 17.04 [9.06, 24.80] & -4.39 [-5.89, -2.89] & 0.40 [0.27, 0.53]\\
\cellcolor{gray!10}{NS} & \cellcolor{gray!10}{Merluccius\_merluccius} & \cellcolor{gray!10}{66.52 [52.16, 80.39]} & \cellcolor{gray!10}{-57.53 [-72.79, -43.53]} & \cellcolor{gray!10}{9.64 [7.65, 11.72]} & \cellcolor{gray!10}{0.12 [-0.01, 0.24]}\\
NS & Microstomus\_kitt & -11.87 [-20.65, -4.00] & 18.15 [10.73, 25.58] & -2.39 [-3.42, -1.46] & 0.28 [0.16, 0.38]\\
\cellcolor{gray!10}{NS} & \cellcolor{gray!10}{Pleuronectes\_platessa} & \cellcolor{gray!10}{19.93 [6.71, 32.68]} & \cellcolor{gray!10}{-48.88 [-64.61, -33.69]} & \cellcolor{gray!10}{3.48 [2.21, 4.80]} & \cellcolor{gray!10}{0.25 [0.13, 0.37]}\\
NS & Pollachius\_virens & 24.45 [12.67, 36.07] & -29.37 [-41.02, -17.82] & 0.34 [-1.25, 1.90] & 0.24 [0.13, 0.36]\\
\cellcolor{gray!10}{NS} & \cellcolor{gray!10}{Scyliorhinus\_canicula} & \cellcolor{gray!10}{-69.47 [-108.66, -36.29]} & \cellcolor{gray!10}{15.64 [6.55, 24.61]} & \cellcolor{gray!10}{-6.82 [-9.18, -4.45]} & \cellcolor{gray!10}{0.46 [0.31, 0.63]}\\
NS & Trisopterus\_esmarkii & 14.81 [6.85, 22.27] & -28.64 [-38.75, -18.71] & 1.02 [-0.11, 2.27] & 0.27 [0.17, 0.37]\\
\cellcolor{gray!10}{NS} & \cellcolor{gray!10}{Trisopterus\_minutus} & \cellcolor{gray!10}{101.46 [63.84, 146.23]} & \cellcolor{gray!10}{-23.03 [-34.24, -12.41]} & \cellcolor{gray!10}{9.38 [6.08, 12.69]} & \cellcolor{gray!10}{-0.06 [-0.25, 0.11]}\\
CBS & Callionymus\_lyra & -9.82 [-28.61, 9.96] & -2.02 [-10.78, 5.81] & 0.83 [-0.31, 2.02] & 0.12 [0.05, 0.19]\\
\cellcolor{gray!10}{CBS} & \cellcolor{gray!10}{Capros\_aper} & \cellcolor{gray!10}{16.71 [-35.29, 70.17]} & \cellcolor{gray!10}{-1.08 [-12.83, 8.73]} & \cellcolor{gray!10}{0.99 [-0.47, 2.77]} & \cellcolor{gray!10}{0.09 [-0.00, 0.18]}\\
CBS & Chelidonichthys\_cuculus & 82.78 [47.97, 120.81] & 12.99 [5.54, 20.88] & -0.71 [-2.83, 0.87] & 0.02 [-0.08, 0.11]\\
\cellcolor{gray!10}{CBS} & \cellcolor{gray!10}{Conger\_conger} & \cellcolor{gray!10}{-47.87 [-77.19, -18.63]} & \cellcolor{gray!10}{2.30 [-7.87, 12.98]} & \cellcolor{gray!10}{-0.01 [-2.08, 1.74]} & \cellcolor{gray!10}{0.17 [0.09, 0.26]}\\
CBS & Eutrigla\_gurnardus & -37.12 [-62.46, -10.67] & -0.02 [-8.01, 7.96] & 0.68 [-0.58, 2.17] & 0.16 [0.08, 0.23]\\
\cellcolor{gray!10}{CBS} & \cellcolor{gray!10}{Gadiculus\_argenteus} & \cellcolor{gray!10}{67.11 [-18.52, 96.18]} & \cellcolor{gray!10}{0.40 [-9.92, 9.97]} & \cellcolor{gray!10}{0.96 [-1.01, 3.38]} & \cellcolor{gray!10}{0.06 [-0.03, 0.15]}\\
CBS & Gadus\_morhua & 105.90 [65.04, 151.03] & 9.94 [2.09, 18.59] & 0.43 [-1.13, 2.15] & 0.01 [-0.09, 0.09]\\
\cellcolor{gray!10}{CBS} & \cellcolor{gray!10}{Glyptocephalus\_cynoglossus} & \cellcolor{gray!10}{-32.98 [-57.42, -5.32]} & \cellcolor{gray!10}{1.35 [-6.68, 9.00]} & \cellcolor{gray!10}{0.08 [-2.04, 1.65]} & \cellcolor{gray!10}{0.15 [0.08, 0.24]}\\
CBS & Helicolenus\_dactylopterus & 14.86 [-17.16, 45.74] & 6.18 [-2.32, 15.33] & -0.07 [-2.23, 1.50] & 0.10 [0.03, 0.17]\\
\cellcolor{gray!10}{CBS} & \cellcolor{gray!10}{Hippoglossoides\_platessoides} & \cellcolor{gray!10}{-32.01 [-59.22, -2.36]} & \cellcolor{gray!10}{0.65 [-6.41, 7.22]} & \cellcolor{gray!10}{0.20 [-1.02, 1.28]} & \cellcolor{gray!10}{0.16 [0.09, 0.24]}\\
CBS & Lepidorhombus\_whiffiagonis & 24.93 [-10.94, 59.66] & 12.29 [4.72, 20.31] & 0.04 [-1.35, 1.60] & 0.09 [0.01, 0.17]\\
\cellcolor{gray!10}{CBS} & \cellcolor{gray!10}{Leucoraja\_naevus} & \cellcolor{gray!10}{13.15 [-22.79, 48.70]} & \cellcolor{gray!10}{2.51 [-4.59, 9.49]} & \cellcolor{gray!10}{0.44 [-0.85, 1.72]} & \cellcolor{gray!10}{0.10 [0.02, 0.16]}\\
CBS & Limanda\_limanda & 7.99 [-13.46, 29.64] & 1.73 [-6.69, 10.21] & 0.91 [-0.21, 2.28] & 0.11 [0.04, 0.18]\\
\cellcolor{gray!10}{CBS} & \cellcolor{gray!10}{Lophius\_budegassa} & \cellcolor{gray!10}{28.20 [-6.25, 62.15]} & \cellcolor{gray!10}{7.26 [-2.73, 17.59]} & \cellcolor{gray!10}{-0.34 [-2.23, 1.10]} & \cellcolor{gray!10}{0.08 [-0.00, 0.15]}\\
CBS & Lophius\_piscatorius & -10.88 [-37.51, 17.28] & -0.86 [-9.14, 7.11] & 0.60 [-1.12, 2.38] & 0.12 [0.05, 0.20]\\
\cellcolor{gray!10}{CBS} & \cellcolor{gray!10}{Melanogrammus\_aeglefinus} & \cellcolor{gray!10}{-13.24 [-39.73, 15.35]} & \cellcolor{gray!10}{10.44 [2.67, 18.75]} & \cellcolor{gray!10}{-1.23 [-2.60, 0.13]} & \cellcolor{gray!10}{0.13 [0.05, 0.22]}\\
CBS & Merlangius\_merlangus & -14.41 [-40.37, 13.96] & -4.34 [-12.08, 3.08] & 1.40 [0.13, 2.92] & 0.13 [0.05, 0.21]\\
\cellcolor{gray!10}{CBS} & \cellcolor{gray!10}{Merluccius\_merluccius} & \cellcolor{gray!10}{35.37 [7.52, 60.72]} & \cellcolor{gray!10}{-0.19 [-10.40, 8.54]} & \cellcolor{gray!10}{1.21 [-0.38, 3.69]} & \cellcolor{gray!10}{0.08 [0.00, 0.15]}\\
CBS & Micromesistius\_poutassou & 113.30 [57.44, 168.10] & 1.19 [-14.34, 13.73] & 1.48 [-0.74, 4.70] & -0.02 [-0.14, 0.08]\\
\cellcolor{gray!10}{CBS} & \cellcolor{gray!10}{Microstomus\_kitt} & \cellcolor{gray!10}{-20.18 [-38.20, -2.17]} & \cellcolor{gray!10}{0.18 [-6.68, 6.79]} & \cellcolor{gray!10}{0.56 [-0.43, 1.57]} & \cellcolor{gray!10}{0.14 [0.07, 0.21]}\\
CBS & Molva\_molva & 197.22 [142.34, 242.58] & 10.77 [0.18, 21.37] & 1.10 [-1.49, 4.41] & -0.10 [-0.23, 0.04]\\
\cellcolor{gray!10}{CBS} & \cellcolor{gray!10}{Phycis\_blennoides} & \cellcolor{gray!10}{2.40 [-40.56, 46.96]} & \cellcolor{gray!10}{6.29 [-4.31, 19.83]} & \cellcolor{gray!10}{0.19 [-1.94, 2.35]} & \cellcolor{gray!10}{0.12 [0.04, 0.20]}\\
CBS & Pleuronectes\_platessa & -15.99 [-35.98, 3.61] & 4.23 [-2.63, 11.55] & 0.22 [-0.84, 1.27] & 0.13 [0.07, 0.20]\\
\cellcolor{gray!10}{CBS} & \cellcolor{gray!10}{Raja\_clavata} & \cellcolor{gray!10}{46.96 [19.06, 74.87]} & \cellcolor{gray!10}{2.61 [-6.90, 11.46]} & \cellcolor{gray!10}{0.83 [-0.71, 2.67]} & \cellcolor{gray!10}{0.06 [-0.01, 0.14]}\\
CBS & Raja\_montagui & 9.64 [-13.63, 33.64] & 0.83 [-6.99, 8.15] & 0.58 [-0.45, 1.67] & 0.11 [0.05, 0.19]\\
\cellcolor{gray!10}{CBS} & \cellcolor{gray!10}{Scyliorhinus\_canicula} & \cellcolor{gray!10}{3.51 [-11.14, 19.60]} & \cellcolor{gray!10}{6.96 [-0.06, 14.16]} & \cellcolor{gray!10}{-0.25 [-1.88, 1.06]} & \cellcolor{gray!10}{0.11 [0.04, 0.18]}\\
CBS & Squalus\_acanthias & -21.96 [-58.07, 15.87] & -1.94 [-10.92, 5.92] & 0.60 [-1.05, 2.31] & 0.13 [0.06, 0.21]\\
\cellcolor{gray!10}{CBS} & \cellcolor{gray!10}{Trisopterus\_esmarkii} & \cellcolor{gray!10}{94.89 [50.67, 128.41]} & \cellcolor{gray!10}{3.03 [-6.17, 11.64]} & \cellcolor{gray!10}{1.27 [-0.23, 3.07]} & \cellcolor{gray!10}{0.02 [-0.07, 0.10]}\\
CBS & Trisopterus\_minutus & 24.28 [0.31, 48.80] & -0.25 [-7.13, 6.14] & 1.07 [-0.00, 2.37] & 0.09 [0.02, 0.16]\\
\cellcolor{gray!10}{CBS} & \cellcolor{gray!10}{Zeus\_faber} & \cellcolor{gray!10}{-28.26 [-52.70, -1.74]} & \cellcolor{gray!10}{7.53 [-0.26, 16.34]} & \cellcolor{gray!10}{-0.24 [-1.68, 1.02]} & \cellcolor{gray!10}{0.15 [0.07, 0.23]}\\
BAL & Enchelyopus\_cimbrius & -0.12 [-11.86, 10.42] & -18.09 [-35.31, 0.68] & -0.83 [-2.35, 0.36] & 0.30 [0.07, 0.48]\\
\cellcolor{gray!10}{BAL} & \cellcolor{gray!10}{Gadus\_morhua} & \cellcolor{gray!10}{-10.17 [-16.00, -4.22]} & \cellcolor{gray!10}{-16.77 [-32.22, -1.62]} & \cellcolor{gray!10}{-0.33 [-1.72, 1.01]} & \cellcolor{gray!10}{0.39 [0.20, 0.68]}\\
BAL & Limanda\_limanda & -21.04 [-28.97, -12.64] & -0.26 [-8.27, 7.50] & 0.08 [-0.77, 0.93] & 0.31 [0.16, 0.48]\\
\cellcolor{gray!10}{BAL} & \cellcolor{gray!10}{Merlangius\_merlangus} & \cellcolor{gray!10}{-4.42 [-20.08, 10.67]} & \cellcolor{gray!10}{7.97 [-1.89, 17.99]} & \cellcolor{gray!10}{0.12 [-1.12, 1.39]} & \cellcolor{gray!10}{0.24 [0.04, 0.40]}\\
BAL & Myoxocephalus\_scorpius & -46.26 [-58.59, -33.38] & 37.77 [6.88, 61.31] & 0.84 [-0.38, 2.09] & 0.27 [0.04, 0.48]\\
\cellcolor{gray!10}{BAL} & \cellcolor{gray!10}{Platichthys\_flesus} & \cellcolor{gray!10}{-45.94 [-55.36, -35.19]} & \cellcolor{gray!10}{-55.18 [-74.84, -35.15]} & \cellcolor{gray!10}{-0.29 [-1.78, 1.20]} & \cellcolor{gray!10}{0.44 [0.24, 0.67]}\\
BAL & Pleuronectes\_platessa & -53.29 [-60.29, -46.14] & 22.60 [9.98, 34.71] & 1.53 [0.13, 3.05] & 0.15 [-0.12, 0.36]\\
\cellcolor{gray!10}{BAL} & \cellcolor{gray!10}{Scophthalmus\_maximus} & \cellcolor{gray!10}{-20.26 [-27.24, -13.52]} & \cellcolor{gray!10}{-36.42 [-54.35, -19.18]} & \cellcolor{gray!10}{-0.88 [-2.02, 0.11]} & \cellcolor{gray!10}{0.46 [0.27, 0.69]}\\
NIC & Arnoglossus\_laterna & -5.01 [-9.58, -1.06] & -21.69 [-32.67, -10.35] & 0.17 [-0.53, 0.85] & 0.02 [-0.03, 0.09]\\
\cellcolor{gray!10}{NIC} & \cellcolor{gray!10}{Cepola\_macrophthalma} & \cellcolor{gray!10}{-0.84 [-4.29, 2.72]} & \cellcolor{gray!10}{-4.89 [-15.75, 5.96]} & \cellcolor{gray!10}{0.17 [-0.55, 0.97]} & \cellcolor{gray!10}{0.02 [-0.04, 0.07]}\\
NIC & Chelidonichthys\_cuculus & -0.72 [-3.24, 1.93] & -2.18 [-17.18, 12.55] & 0.12 [-0.66, 0.90] & 0.02 [-0.04, 0.07]\\
\cellcolor{gray!10}{NIC} & \cellcolor{gray!10}{Chelidonichthys\_lucerna} & \cellcolor{gray!10}{0.57 [-3.61, 5.27]} & \cellcolor{gray!10}{12.73 [-5.12, 31.85]} & \cellcolor{gray!10}{-0.09 [-1.02, 0.55]} & \cellcolor{gray!10}{0.01 [-0.05, 0.07]}\\
NIC & Conger\_conger & -3.63 [-7.06, -0.46] & -11.85 [-25.84, 2.04] & 0.16 [-0.78, 1.18] & 0.02 [-0.03, 0.08]\\
\cellcolor{gray!10}{NIC} & \cellcolor{gray!10}{Helicolenus\_dactylopterus} & \cellcolor{gray!10}{-1.91 [-7.79, 3.10]} & \cellcolor{gray!10}{-1.66 [-17.56, 16.12]} & \cellcolor{gray!10}{0.14 [-0.88, 1.27]} & \cellcolor{gray!10}{0.02 [-0.04, 0.08]}\\
NIC & Lepidorhombus\_boscii & -5.89 [-10.34, -1.73] & -21.01 [-33.10, -9.35] & 0.29 [-0.60, 1.55] & 0.02 [-0.03, 0.10]\\
\cellcolor{gray!10}{NIC} & \cellcolor{gray!10}{Lepidorhombus\_whiffiagonis} & \cellcolor{gray!10}{0.45 [-2.44, 3.39]} & \cellcolor{gray!10}{-20.95 [-36.73, -4.78]} & \cellcolor{gray!10}{0.61 [-0.20, 2.11]} & \cellcolor{gray!10}{0.02 [-0.04, 0.09]}\\
NIC & Lophius\_budegassa & 0.67 [-3.60, 5.27] & 21.57 [1.62, 43.49] & -0.31 [-2.00, 0.51] & 0.01 [-0.06, 0.08]\\
\cellcolor{gray!10}{NIC} & \cellcolor{gray!10}{Merluccius\_merluccius} & \cellcolor{gray!10}{-3.04 [-8.10, 0.84]} & \cellcolor{gray!10}{3.00 [-12.21, 18.50]} & \cellcolor{gray!10}{-0.04 [-1.06, 0.67]} & \cellcolor{gray!10}{0.02 [-0.04, 0.07]}\\
NIC & Microchirus\_variegatus & -3.42 [-7.83, 0.34] & -10.45 [-23.39, 2.95] & 0.16 [-0.69, 1.05] & 0.02 [-0.04, 0.08]\\
\cellcolor{gray!10}{NIC} & \cellcolor{gray!10}{Micromesistius\_poutassou} & \cellcolor{gray!10}{0.87 [-3.21, 6.21]} & \cellcolor{gray!10}{-0.06 [-17.06, 17.46]} & \cellcolor{gray!10}{0.31 [-0.49, 2.15]} & \cellcolor{gray!10}{0.01 [-0.06, 0.06]}\\
NIC & Pagellus\_acarne & -0.31 [-3.01, 2.51] & 6.26 [-13.22, 28.57] & 0.08 [-0.96, 1.06] & 0.01 [-0.05, 0.07]\\
\cellcolor{gray!10}{NIC} & \cellcolor{gray!10}{Raja\_clavata} & \cellcolor{gray!10}{-0.59 [-3.15, 2.08]} & \cellcolor{gray!10}{-11.83 [-26.11, 1.71]} & \cellcolor{gray!10}{0.32 [-0.38, 1.41]} & \cellcolor{gray!10}{0.02 [-0.04, 0.07]}\\
NIC & Raja\_montagui & -0.40 [-2.88, 2.17] & -2.89 [-14.86, 9.37] & 0.18 [-0.58, 1.04] & 0.02 [-0.04, 0.07]\\
\cellcolor{gray!10}{NIC} & \cellcolor{gray!10}{Scyliorhinus\_canicula} & \cellcolor{gray!10}{-2.42 [-5.30, 0.24]} & \cellcolor{gray!10}{-9.08 [-24.20, 5.93]} & \cellcolor{gray!10}{0.02 [-1.01, 0.69]} & \cellcolor{gray!10}{0.02 [-0.03, 0.08]}\\
NIC & Trisopterus\_luscus & -2.66 [-10.15, 2.61] & -0.43 [-22.30, 23.44] & 0.10 [-1.00, 1.24] & 0.02 [-0.04, 0.08]\\
\cellcolor{gray!10}{NIC} & \cellcolor{gray!10}{Trisopterus\_minutus} & \cellcolor{gray!10}{0.53 [-2.48, 3.96]} & \cellcolor{gray!10}{-0.35 [-14.46, 13.76]} & \cellcolor{gray!10}{0.17 [-0.72, 1.21]} & \cellcolor{gray!10}{0.01 [-0.05, 0.07]}\\
NIC & Zeus\_faber & -0.63 [-3.63, 2.44] & 4.68 [-9.89, 20.24] & 0.05 [-0.79, 0.80] & 0.01 [-0.05, 0.06]\\*
\end{longtable}
\endgroup{}

\clearpage

\renewcommand{\thetable}{A\arabic{table}}
\setcounter{table}{0}
\renewcommand{\thefigure}{A\arabic{figure}}
\setcounter{figure}{0}
\renewcommand{\theequation}{A\arabic{equation}}
\setcounter{equation}{0}

\renewcommand{\theHfigure}{A\arabic{figure}}
\renewcommand{\theHtable}{A\arabic{table}}
\renewcommand{\theHequation}{A\arabic{equation}}

\begin{bibunit} % for appendix
\subsection*{Appendix A - Additional information on spatiotemporal models}\label{sec:A}

The general form of the spatiotemporal GLMM can be represented as:
\begin{align}
    y_{s,t} &\sim \text{Tweedie} \left( \mu_{\mathbf{s},t}, p, \phi \right), \quad 1 < p < 2, \label{eq:1} \\
    \mu_{s,t} &= \exp \left(\boldsymbol{\beta} \boldsymbol{\text{X}_{\boldsymbol{s},t}}  + \omega_{\boldsymbol{s}} + \epsilon_{\boldsymbol{s},t} \right), \label{eq:2} \\
    \boldsymbol{\omega} &\sim \operatorname{MVN} \left(\boldsymbol{0}, \boldsymbol{\Sigma}_{\omega} \right), \label{eq:3} \\
    \boldsymbol{\epsilon}_{t} &\sim \operatorname{MVN} \left(0, \boldsymbol{\Sigma}_{\epsilon} \right), \label{eq:4}
\end{align}
where \( y_{\mathbf{s},t} \) represents fish density (kg/km\(^2\)) at spatial location \( \mathbf{s} \) and time \( t \), \( \mu \) is the expected mean density, and \( p \) and \( \phi \) are the power and dispersion parameters of the Tweedie distribution, respectively. \( \mathbf{X}_{\mathbf{s},t} \) is the design matrix of covariates including year (as a categorical variable) and a second-degree polynomial of log(depth), which helps constrain predictions to biologically plausible magnitudes across the depth gradient. The vector \( \boldsymbol{\beta} \) contains the corresponding fixed-effect coefficients.

Because fish species are not constrained by survey boundaries, and some surveys cover only a small part of the distribution range of species we aggregated multiple surveys into broader regions for modeling when appropriate. Surveys with extensive and clearly defined coverage (e.g., the North Sea IBTS) were modeled independently, while those with contiguous spatial domains and partial overlap in time and space (i.e., surveys conducted in British Columbia, Northeast U.S \& Scotian Shelf and Celtic-Biscay shelf) were combined (Fig \ref{fig:survey}). For regions with multiple surveys, we included survey identity as a fixed effect to control for methodological differences. We did not apply this correction in British Columbia because surveys there follow harmonized field protocols and are typically analysed jointly \citep[e.g.,][]{Anderson2019BCGroundfish, ward_win_2024}. In regions where surveys occur multiple times per year (BAL, NS, NEUS-SS, CBS, GOM), we included quarter as a fixed effect to account for within-year variation in sampling. In cases where a designated sampling period extended across multiple quarters, we used combined categories (e.g., Q2–Q3 or Q3-Q4).

The parameters \( \omega_{\mathbf{s}} \) and \( \epsilon_{\mathbf{s},t} \) (Equations \ref{eq:3}--\ref{eq:4}) represent spatial and spatiotemporal random effects, respectively, both modeled as Gaussian Markov random fields \citep{lindgren_explicit_2011}. The spatial random field is shared across years, and represents static spatial effects such as habitat that are not accounted for by the fixed effects, while the spatiotemporal fields represent interannual spatial variation. We assumed spatiotemporal random fields to be independent across years to capture annual variation in spatial structure. While a first-order autoregressive (AR1) process could have been used to model temporal dependence, we opted for the independent formulation to strike a balance between computational efficiency and model flexibility. This approach allows spatial patterns to vary freely from year to year, accommodating potential non-stationary dynamics in species distributions without imposing strict temporal correlation structures. For British Columbia surveys, which have a biennial sampling structure, we modeled spatiotemporal variation as a random walk process. This modifies Equation \ref{eq:4} into $\epsilon_{\boldsymbol{s},t} \sim  \operatorname{MVN}(\epsilon_{t-1},\boldsymbol{\Sigma}_{\epsilon})$ to allow for flexibility in estimating the spatial and temporal processes in years without data \citep{ward_win_2024}.

Latent spatial and spatiotemporal random fields were approximated using a triangulated mesh with a minimum spacing (cutoff) of 20 km constructed with the \texttt{fmesher} R package \citep{lindgren_fmesher_2025}. This cutoff distance represents the smallest allowed distance between two mesh vertices. We assumed a shared range parameter (distance at which points are effectively independent \citep{lindgren_explicit_2011}) between the spatial and spatiotemporal fields, while allowing each field to have its own variance.

Model parameters were estimated by maximizing the marginal log-likelihood using Template Model Builder (TMB; \citep{kristensen_tmb_2016}, which applies the Laplace approximation to integrate out random effects. Models were fit in R 4.1.0 \citep{r_core_team_r_2021} using the \texttt{sdmTMB} package \citep{anderson_sdmtmb_2024}, which interfaces TMB with INLA-based spatial methods \citep{rue_approximate_2009}. Only models with a positive-definite Hessian and a maximum absolute log-likelihood gradient $< 0.001$ were considered successfully converged and used for inference \citep{anderson_sdmtmb_2024}.

\begin{figure}[H]
    \centering
    \includegraphics[width=1\linewidth]{output/figures/supp/surveys_supp.png}
    \caption{(a) Regions where multiple surveys were combined. (b) Overlap in survey months within each region.}
    \label{fig:survey}
\end{figure}

\putbib

\end{bibunit}

\newpage

\renewcommand{\thetable}{B\arabic{table}}
\setcounter{table}{0}
\renewcommand{\thefigure}{B\arabic{figure}}
\setcounter{figure}{0}
\renewcommand{\theequation}{B\arabic{equation}}
\setcounter{equation}{0}

\renewcommand{\theHfigure}{B\arabic{figure}}
\renewcommand{\theHtable}{B\arabic{table}}
\renewcommand{\theHequation}{B\arabic{equation}}

\begin{bibunit}

\subsection*{Appendix B - Additional information on Bayesian trend analysis}\label{sec:B}

\noindent
\textbf{Rationale for the Student-t likelihood}.
We first modeled spatial and thermal responses with a multivariate normal (MVN) likelihood because these metrics often covary (e.g., depth shifts co-occurring with changes in geographic range centroids). Posterior predictive checks, however, showed heavier tails than the MVN could capture. We therefore modeled each response with a univariate Student-t likelihood, which better accommodates extreme values. We preserved correlations among responses by modeling group-level effects with a MVN covariance structure, allowing shared variation across responses to be captured explicitly. This specification improved predictive performance, as shown by higher expected log predictive density (ELPD) from leave-one-out cross-validation compared with the MVN model (Table~\ref{tab:elpd}; \citep{vehtari_practical_2017}).

\begin{table}[!htbp]
\centering
    \caption{Expected log predictive density (ELPD) differences between models. ELPD differences are reported relative to the best-performing model, which is assigned a value of zero. Negative values indicate poorer expected out-of-sample predictive performance. SE denotes the standard error of the ELPD difference}
    \label{tab:elpd}
\begin{table}

<<<<<<< HEAD
\caption{\label{tab:tab:elpd}Model comparison based on leave-one-out cross-validation (LOO).  MVN denotes a multivariate normal distribution, while Student-t $+$ MVN specifies a univariate Student-t distribution for each response combined with multivariate normal random effects.
=======
\caption{\label{tab:elpd}Model comparison based on leave-one-out cross-validation (LOO).  MVN denotes a multivariate normal distribution, while Student-t $+$ MVN specifies a univariate Student-t distribution for each response combined with multivariate normal random effects.
>>>>>>> 339f4f2 (added label)
  The expected log predictive density (ELPD) reflects out-of-sample predictive accuracy, with higher values indicating better performance. ELPD differences are shown relative to the best-fitting model, so negative values indicate worse fit. The standard error (SE) of the difference reflects uncertainty; differences large relative to SE suggest meaningful performance gaps.}
\centering
\begin{tabular}[t]{lrr}
\toprule
Model & ELPD difference & SE difference\\
\midrule
Student-t $+$ MVN & 0.0 & 0.00\\
MVN & -169139.8 & 11054.05\\
\bottomrule
\end{tabular}
\end{table}

\end{table}

\noindent
\textbf{Hierarchical structure.} 
An alternative to our approach would have been to fit separate models for each species and response. Such models, however, ignore information shared across taxa, regions, and outcomes. By using a hierarchical structure, we applied partial pooling: estimates borrow strength from the full dataset while still allowing region- and species-specific deviations. This shrinkage yields more stable estimates than independent fits (Fig.~\ref{fig:shrinkage_supp}; \citep{mcelreath_statistical_2018}) and allows us to estimate global, regional, and species-level trends within a single unified framework. We modeled correlations among response variables at the regional level because species within the same region experience similar environmental conditions and constraints. This structure captures shared variation in temporal trends among species within regions while allowing species-specific departures from regional patterns.

\begin{figure}[!htbp]
    \centering
    \includegraphics[width=1\linewidth]{output/figures/supp/shrinkage_supp.png}
    \caption{Comparison of hierarchical trend estimates versus separate linear trend fits. Posterior slopes from the full hierarchical model (red) are compared to slopes obtained from independent linear models (blue) fitted to a random subset of 50 species. Partial pooling in the hierarchical model shrinks extreme estimates toward the regional and global mean, producing more stable and biologically plausible trend estimates across species.}
    \label{fig:shrinkage_supp}
\end{figure}

\noindent
\textbf{Priors specification and validation}.
All priors are summarized in Table~\ref{tab:priors_supp}. Throughout, Normal$(\mu,\sigma)$ and Student-t$(\nu,\mu,\sigma)$ denote distributions with location $\mu$ and dispersion $\sigma$, where $\sigma$ is the standard deviation for the Normal and the scale parameter for the Student-t, and $\nu$ denotes the degrees of freedom. We specified weakly informative priors for global temporal slopes (decade effects) based on published rates of range shifts, depth changes, and ocean warming (Fig.~\ref{fig:priors_supp}a). For latitudinal and longitudinal centroids, we used Normal$(0,50)$ km decade$^{-1}$ priors, accommodating observed shifts of up to ~30 km decade$^{-1}$ in marine taxa \citep{poloczanska_global_2013}. For depth, we applied a Normal$(0,5)$ m decade$^{-1}$ prior, consistent with reported deepening of ~3.6 m decade$^{-1}$ in North Sea demersal fish \citep{dulvy_climate_2008}. For thermal niches, we used a Normal$(0,0.5)$ $^\circ$C decade$^{-1}$ prior, encompassing rates observed in rapidly warming shelf seas \citep{chen_longterm_2020}.
To model group-level variation in slopes, we assigned Student-t distributions truncated to $(0,\infty)$ with three degrees of freedom, allowing substantial but biologically plausible variation. Region-level slope standard deviations followed Student-t$(3,0,30)$ km decade$^{-1}$ priors for spatial centroids, Student-t$(3,0,10)$ m decade$^{-1}$ for depth, and Student-t$(3,0,0.2)$ $^\circ$C decade$^{-1}$ for thermal niches. For species nested within regions, we specified broader priors---Student-t$(3,0,40)$ km decade$^{-1}$, Student-t$(3,0,20)$ m decade$^{-1}$, and Student-t$(3,0,0.4)$ $^\circ$C decade$^{-1}$, respectively.
These priors were consistent with the empirical variability observed in the data, providing appropriate regularization while remaining flexible enough to capture biologically plausible trends (Fig.~\ref{fig:priors_supp}b–c). All remaining model parameters retained the default priors provided by \texttt{brms} (\citep{burkner_brms_2017}; Table~\ref{tab:priors_supp}).

\begin{table}[!htbp]
\centering
    \caption{Prior distributions for model parameters. Fixed effects ($\beta$), group-level and standard deviations ($\sigma$), degrees of freedom ($\nu$) for the Student-t distribution, and the correlation matrix of region-level effects ($\mathbf{R}_r$). The source column indicates whether the prior was changed from the default.}
    \label{tab:priors_supp}

\begin{tabular}{llllll}
\toprule
Prior & Parameter & Coefficient & Group & Response & Source\\
\midrule
LKJ(1) & $\mathbf{R}_{r}$ &  &  &  & default\\
Normal(0, 50) & $\beta$ & decade &  & lat centroid & user\\
Normal(0, 50) & $\beta$ & decade &  & lon centroid & user\\
Normal(0, 5) & $\beta$ & decade &  & depth niche & user\\
Normal(0, 0.5) & $\beta$ & decade &  & thermal niche & user\\
Gamma(2, 0.1) & $\nu$ &  &  & lat centroid & default\\
Gamma(2, 0.1) & $\nu$ &  &  & lon centroid & default\\
Gamma(2, 0.1) & $\nu$ &  &  & depth niche & default\\
Gamma(2, 0.1) & $\nu$ &  &  & thermal niche & default\\
Student-t(3, 0, 31) & $\sigma$ &  &  & lat centroid & default\\
Student-t(3, 0, 27.9) & $\sigma$ &  &  & lon centroid & default\\
Student-t(3, 0, 3.9) & $\sigma$ &  &  & depth niche & default\\
Student-t(3, 0, 2.5) & $\sigma$ &  &  & thermal niche & default\\
Student-t(3, 0, 30) & $\sigma$ & decade & region & lat centroid & user\\
Student-t(3, 0, 30) & $\sigma$ & decade & region & lon centroid & user\\
Student-t(3, 0, 10) & $\sigma$ & decade & region & depth niche & user\\
Student-t(3, 0, 0.2) & $\sigma$ & decade & region & thermal niche & user\\
Student-t(3, 0, 40) & $\sigma$ & decade & region:species & lat centroid & user\\
Student-t(3, 0, 40) & $\sigma$ & decade & region:species & lon centroid & user\\
Student-t(3, 0, 20) & $\sigma$ & decade & region:species & depth niche & user\\
Student-t(3, 0, 0.4) & $\sigma$ & decade & region:species & thermal niche & user\\
Student-t(3, 0, 31) & $\sigma$ &  &  & lat centroid & default\\
Student-t(3, 0, 27.9) & $\sigma$ &  &  & lon centroid & default\\
Student-t(3, 0, 3.9) & $\sigma$ &  &  & depth niche & default\\
Student-t(3, 0, 2.5) & $\sigma$ &  &  & thermal niche & default\\
\bottomrule
\end{tabular}

\end{table}


\begin{figure}[!htbp]
    \centering
    \includegraphics[width=1\linewidth]{output/figures/supp/priors_supp.png}
    \caption{Prior specification and validation for temporal trend parameters.
(a) Priors for global temporal slopes (decade effects) shown relative to literature-reported values (black dashed lines).
(b) Priors for region-level slope variation compared with empirical distributions from the data (orange dashed lines).
(c) Priors for species-level slope variation nested within regions compared with empirical distributions (orange dashed lines). Priors are weakly informative and consistent with observed variability, providing regularization without constraining plausible trends.}
    \label{fig:priors_supp}
\end{figure}


\noindent
\textbf{Model diagnostics}.
Model fit and convergence were assessed using posterior predictive checks (PPCs) and standard MCMC diagnostics. PPCs confirmed that the model captures the central tendency, variability, and distributional shape of the observed data (Fig.~\ref{fig:pp_supp}). Convergence diagnostics indicated $\hat{R} < 1.01$ for all parameters, no divergent transitions, and effective sample sizes $> 400$, confirming reliable posterior sampling (Fig.~\ref{fig:rhat_supp}; \cite{vehtari_rank-normalization_2021}). One parameter fell slightly below the recommended bulk ESS threshold, but visual inspection of the chains indicated satisfactory mixing (not shown), supporting reliable posterior sampling.

\begin{figure}[htbp]
    \centering
    \includegraphics[width=1\linewidth]{output/figures/supp/pp_supp.png}
    \caption{Posterior predictive checks for the four outcomes. Observed values ($y$) are compared with simulated datasets drawn from the posterior predictive distribution ($y_{rep}$). Close alignment between the two indicates good model fit.}
    \label{fig:pp_supp}
\end{figure}

\begin{figure}[htbp]
    \centering
    \includegraphics[width=0.7\linewidth]{output/figures/supp/rhat_supp.png}
    \caption{Convergence diagnostics for the hierarchical trend model. Panels show $\hat{R}$, bulk effective sample size (ESS), and tail ESS for all parameters. Yellow regions indicate values outside recommended thresholds.}
    \label{fig:rhat_supp}
\end{figure}



 % same .bib, reused keys get prefix
\clearpage
\putbib

\end{bibunit}

%\begin{table}

\caption{\label{tab:exp_obs_region}Correlation ($
ho$) and directional agreement between observed species shifts and those expected under thermal envelope tracking. The proportion of aligned responses indicates the fraction of species whose observed and expected shifts occurred in the same direction (i.e., shared the same sign). N denotes the number of species compared for each shift type within each region.}
\centering
\begin{tabular}[t]{llrrrr}
\toprule
Region & Shift type & Proportion of aligned responses & N & $\rho$ & $p$-value\\
\midrule
North Sea (NS) & Latitudinal & 0.64 & 11 & 0.90 & 0.000\\
North Sea (NS) & Depth & 0.64 & 11 & 0.87 & 0.000\\
North Sea (NS) & Longitudinal & 0.82 & 11 & 0.85 & 0.001\\
North Iberian Coast (NIC) & Latitudinal & 0.89 & 9 & 0.81 & 0.008\\
Barents Sea (BS) & Latitudinal & 0.50 & 12 & 0.65 & 0.021\\
\addlinespace
East Bering Sea (EBS) & Longitudinal & 0.50 & 8 & 0.51 & 0.190\\
North Iberian Coast (NIC) & Longitudinal & 0.67 & 9 & 0.47 & 0.200\\
East Bering Sea (EBS) & Latitudinal & 0.88 & 8 & 0.42 & 0.300\\
U.S. West Coast (USWC) & Latitudinal & 0.42 & 19 & 0.42 & 0.075\\
NE U.S. & Scotian Shelf (NEUS–SS) & Longitudinal & 0.57 & 14 & 0.34 & 0.240\\
\addlinespace
U.S. West Coast (USWC) & Depth & 0.79 & 19 & 0.34 & 0.150\\
Baltic Sea (BAL) & Longitudinal & 0.60 & 5 & 0.33 & 0.590\\
Barents Sea (BS) & Depth & 0.50 & 12 & 0.32 & 0.310\\
British Columbia (BC) & Depth & 0.67 & 24 & 0.19 & 0.370\\
British Columbia (BC) & Longitudinal & 0.38 & 24 & 0.16 & 0.470\\
\addlinespace
Celtic–Biscay Shelf (CBS) & Latitudinal & 0.45 & 22 & 0.14 & 0.550\\
U.S. West Coast (USWC) & Longitudinal & 0.42 & 19 & 0.13 & 0.600\\
East Bering Sea (EBS) & Depth & 0.25 & 8 & 0.11 & 0.790\\
British Columbia (BC) & Latitudinal & 0.38 & 24 & 0.09 & 0.690\\
Gulf of Mexico (GOM) & Latitudinal & 0.44 & 18 & 0.06 & 0.830\\
\addlinespace
Gulf of Mexico (GOM) & Depth & 0.44 & 18 & 0.06 & 0.820\\
Gulf of Alaska (GOA) & Longitudinal & 0.47 & 15 & 0.00 & 0.990\\
Gulf of Alaska (GOA) & Depth & 0.67 & 15 & -0.03 & 0.910\\
Baltic Sea (BAL) & Depth & 0.40 & 5 & -0.05 & 0.940\\
Gulf of Alaska (GOA) & Latitudinal & 0.60 & 15 & -0.08 & 0.780\\
\addlinespace
NE U.S. & Scotian Shelf (NEUS–SS) & Latitudinal & 0.57 & 14 & -0.08 & 0.790\\
NE U.S. & Scotian Shelf (NEUS–SS) & Depth & 0.79 & 14 & -0.10 & 0.740\\
Celtic–Biscay Shelf (CBS) & Longitudinal & 0.50 & 22 & -0.12 & 0.610\\
Barents Sea (BS) & Longitudinal & 0.33 & 12 & -0.17 & 0.600\\
Celtic–Biscay Shelf (CBS) & Depth & 0.55 & 22 & -0.21 & 0.350\\
\addlinespace
Gulf of Mexico (GOM) & Longitudinal & 0.50 & 18 & -0.27 & 0.280\\
Baltic Sea (BAL) & Latitudinal & 0.40 & 5 & -0.50 & 0.390\\
North Iberian Coast (NIC) & Depth & 0.33 & 9 & -0.81 & 0.008\\
\bottomrule
\end{tabular}
\end{table}



\end{document}
